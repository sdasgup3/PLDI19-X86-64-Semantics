%
% The first command in your LaTeX source must be the \documentclass command.
\documentclass[sigplan,screen]{acmart}

%
% \BibTeX command to typeset BibTeX logo in the docs
\AtBeginDocument{%
  \providecommand\BibTeX{{%
    \normalfont B\kern-0.5em{\scshape i\kern-0.25em b}\kern-0.8em\TeX}}}

% Rights management information. 
% This information is sent to you when you complete the rights form.
% These commands have SAMPLE values in them; it is your responsibility as an author to replace
% the commands and values with those provided to you when you complete the rights form.
%
% These commands are for a PROCEEDINGS abstract or paper.
\setcopyright{acmlicensed}
\acmPrice{}
\acmDOI{10.1145/3314221.3314601}
\acmYear{2019}
\copyrightyear{2019}
\acmISBN{978-1-4503-6712-7/19/06}
\acmConference[PLDI '19]{Proceedings of the 40th ACM SIGPLAN Conference on Programming Language Design and Implementation}{June 22--26, 2019}{Phoenix, AZ, USA}
\acmBooktitle{Proceedings of the 40th ACM SIGPLAN Conference on Programming Language Design and Implementation (PLDI '19), June 22--26, 2019, Phoenix, AZ, USA}

%
% These commands are for a JOURNAL article.
%\setcopyright{acmcopyright}
%\acmJournal{TOG}
%\acmYear{2018}\acmVolume{37}\acmNumber{4}\acmArticle{111}\acmMonth{8}
%\acmDOI{10.1145/1122445.1122456}

%
% Submission ID. 
% Use this when submitting an article to a sponsored event. You'll receive a unique submission ID from the organizers
% of the event, and this ID should be used as the parameter to this command.
%\acmSubmissionID{123-A56-BU3}

%
% The majority of ACM publications use numbered citations and references. If you are preparing content for an event
% sponsored by ACM SIGGRAPH, you must use the "author year" style of citations and references. Uncommenting
% the next command will enable that style.
%\citestyle{acmauthoryear}

%% Custom Packages
\usepackage[T1]{fontenc}
\usepackage{listings}
\usepackage{color}
\usepackage{xspace}
\usepackage[title]{appendix}
%\usepackage{tikz}
%\usepackage{pgf-pie}
\usepackage{forest}
\usepackage{amsmath,amssymb}
\usepackage{ctable}
\usepackage{pifont}
\usepackage{calculator}

\setlength{\parskip}{0.5pt plus 2pt minus 3pt}
%\setlength{\textfloatsep}{0\baselineskip plus 0.2\baselineskip minus 0.5\baselineskip}
\newenvironment{tightcenter}{%
    \setlength\topsep{4pt}
    \setlength\parskip{-2pt}
    \begin{center}
    }{%
    \end{center}
}

% FIXME: overleaf broken if uncommented
% \usepackage{tikz-qtree}
% \usetikzlibrary{arrows,shapes,positioning,shadows,trees}
% \usetikzlibrary{shadows,trees}


%% Custom Commands
\def\Code#1{\texttt{#1} }
%\def\Comment#1{}
\def\Comment#1{\textbf{\textsl{\color{red}  $\langle\!\langle$#1$\rangle\!\rangle$}} }
\newcommand{\percentage}[2]{\DIVIDE{#1}{#2}{\duv}\MULTIPLY{\div}{100}{\res}$\res\%$}
\newcommand{\revisit}[1]{{\color{red} Sandeep: #1}}
\newcommand{\Qd}[1]{{\color{red} Daejun: #1}}
\newcommand{\Qt}[1]{{\color{red} Theo: #1}}
\newcommand{\BW}[1]{{\color{red} Borrowed: #1}}
\newcommand{\Added}[1]{{\color{red} #1}}
\newcommand{\SC}[1]{{\color{blue} #1}}
\newcommand{\AEC}[1]{{\color{blue} #1}}
\newcommand{\cmt}[1]{}
\newcommand{\xmark}{{\color{red} \ding{55}}}
\newcommand{\cmark}{{\color{green} \ding{51}}}
\newcommand{\ISA}{x86-64\xspace}
%\newcommand{\K}{\mbox{$\mathbb{K}$}\xspace}
\newcommand{\Z}{$\mathbb{Z}3$\xspace}
\newcommand{\uif}{uninterpreted functions}
\newcommand{\Strata}{Strata\xspace}
\newcommand{\Stoke}{Stoke\xspace}
\newcommand{\initS}{{\tt initial search}}
\newcommand{\secS}{{\tt secondary searches}}
%\newcommand{\K}{\ensuremath{\mathcal{{\tt K}}}\xspace}
\newcommand{\TS}[1]{{\tt #1}}
%\newcommand{\instr}[1]{\texttt{#1}}
\newcommand{\instr}[1]{\textbf{\color{brown}\m{#1}}}
\newcommand{\reg}[1]{\s{\%#1}}
\newcommand{\mem}[2]{\s{#1(\%#2)}}
\newcommand{\opcode}[1]{\ensuremath{#1}}
%\newcommand{\cond}[1]{\ensuremath{#1}}
\newcommand{\extract}{\emph{extract}\xspace}
\newcommand{\extractMInt}{\emph{extractMInt}\xspace}
\newcommand{\false}{\textbf{False}}
\newcommand{\true}{\textbf{True}}
\newcommand{\bool}{\texttt{Bool}\xspace}
\newcommand{\incfig}[1]{\includegraphics[scale=.7]{#1}}
\newcommand{\CF}[2]{$\s{F}_{\s{#2}}^{\s{#1}}$}
\newcommand{\GN}[2]{$G{[#2]}^{#1}$}
\newcommand{\udef}{\emph{undef}\xspace}
%\newcommand{\bv}[2]{$#1\text{'}#2$\xspace}
\newcommand{\bv}[2]{\s{#1}\text{'}\s{#2}}

\newcommand{\rating}[1]{%
    \begin{tikzpicture}[x=1ex,y=1ex]
    \begin{scope}
    \clip (0,1) circle (1);
    \fill[black] (-1,0) rectangle (1,#1/50);
    \end{scope}
    \draw[black, thin, radius=1] (0,1) circle;
    \end{tikzpicture}%
}

% Current Support
%\newcommand{\currentIS}{$3186$} %including jmp label
\newcommand{\currentIS}{$3155$}
\newcommand{\currentIntel}{$774$}
%\newcommand{\currentManual}{$1199$} %3104 - 1905
%\newcommand{\currentManual}{$1281$} %3186 - 1905
%\newcommand{\currentManualPerc}{$40\%$} %100 - \strataPerc
% Total
\newcommand{\totalIS}{$3736$}
\newcommand{\totalIntel}{$996$}
%\newcommand{\totalIntel}{$1,000$}
\newcommand{\dup}{$109$}
% Strata
\newcommand{\strataIS}{$1796$}
\newcommand{\strataIntel}{$466$}
\newcommand{\strataWithDupIS}{$1905$}
\newcommand{\strataRegVarIS}{$692$}
% Unsupported
\newcommand{\system}{$210$}
\newcommand{\Xmmx}{$336$}
\newcommand{\crypto}{$35$}

\newcommand{\strataPerc}{$47\%$} % 466/996 or  1905 / 3736
\newcommand{\goelPerc}{$33\%$} 
\newcommand{\sailPerc}{$15\%$} 
% Stoke disjoin from Strata
%\newcommand{\stokeIS}{$332$} % 262 + 15 + 9 + 46. ALso 332/3767 == 9%
% 1432(strata common) + 332
\newcommand{\stokeIS}{${\sim}1764$}
%\newcommand{\stokeExcPerc}{$9\%$}
% Strata stoke combined
%\newcommand{\strataPlusStokeIS}{$2237$} % 

%\newcommand{\unsupp}{$939$}

%%%%%% Immediates
%\newcommand{\ImmUg}{$146$} % 118 + 28
%\newcommand{\ImmTotal}{$308$}
%\newcommand{\ImmG}{$190$}

%%%%%%% Registers
%\newcommand{\RegTOTAL}{$1133$} % 1083 + 50
%\newcommand{\RegSTRAT}{$742$} % 692 + 50
%\newcommand{\RegSTOK}{$262$}
%\newcommand{\RegMAN}{$129$}

%%% toture status
\newcommand{\TortureTotal}{$1576$} %
\newcommand{\TortureExclude}{$6$} % 6 + 22
\newcommand{\TortureInclude}{$1548$}
\newcommand{\TortureUifsInstr}{$293$} % 134(all three jobs) +  48
\newcommand{\TortureUifs}{$35$}
\newcommand{\TortureCoverage}{$963$}
%%% Undef counts
\newcommand{\undefTotal}{$474$}
\newcommand{\undefIntel}{$32$}
\newcommand{\undefPerc}{$3$} %32/1000

\PassOptionsToPackage{pdftex,usenames,dvipsnames,svgnames,x11names}{xcolor}
\PassOptionsToPackage{pdftex}{hyperref}
\usepackage[style=math]{k}


% Slightli modified original version of \reduce. Altered baseline for more compactness.
% No support for multiline.
\newcommand{\reduceClassic}[2]{\hbox{%
  \begin{tikzpicture}[baseline=(top.south), %(top.base), - default, less compact
                      inner xsep=0pt,
                      inner ysep=.3333ex,
                      minimum width=2em]
    \path
          % Original version. No support for line wrapping.
          node (top) [inner ysep=1ex]{$#1$ \mathstrut}

          % New version. Line wrapping support.
          %node (top) [inner ysep=1ex]{$ \begin{array}{@{}c@{}} #1 \end{array} $ \mathstrut}
          (top.south)
          % Original version. No support for line wrapping.
          node (bottom) [anchor=north, inner ysep=.5ex] {$#2$};

          % New version. Line wrapping support.
          % Adds a little bit of vertical space, but the difference is truly insignificant. All the experiments below failed to remove it.
          %node (bottom) [anchor=north, inner ysep=.5ex] {$ \begin{array}{@{}c@{}} #2 \end{array} $};
          % no extra effect
          % node (bottom) [anchor=north, inner ysep=.5ex] {\vspace{-1em} $ \begin{array}{@{}c@{}} #2 \end{array} $};
          % trying mathstrut - some horizontal re-alignment, but no vertical
          % node (bottom) [anchor=north, inner ysep=.5ex] {$ \begin{array}{@{}c@{}} #2 \end{array} $ \mathstrut};
          % no outer ysep (if no inner - looks bad)
          %node (bottom) [anchor=north, inner ysep=.5ex, outer ysep=0] {$ \begin{array}{@{}c@{}} #2 \end{array} $};
          % \vskip -1em just don't compile no matter where we put it
    \path[draw,thin,solid] let \p1 = (current bounding box.west),
                               \p2 = (current bounding box.east),
                               \p3 = (top.south)
                           in (\x1,\y3) -- (\x2,\y3);
    % Solid arrow (augmenting the solid line).
    \path[fill] (top.south) ++(2pt,0) -- ++(-4pt,0) -- ++(2pt,-1.5pt) -- cycle;
  \end{tikzpicture}%
}}

% Defalut version of \reduce in this document.
%   Support for multi-line LHS and RHS
%   Good compactness. Separators adjusted to be aligned with \reduceClassic
\newcommand{\reduceMulti}[2]{\hbox{%
  \begin{tikzpicture}[baseline=(top.south), %(top.base), - default, less compact
                      inner xsep=0pt,
                      inner ysep=.3333ex,
                      minimum width=2em]
    \path
          % New version. Line wrapping support.
          node (top) [
            %inner ysep=1ex
            inner ysep=0.6ex
          ]{ $ \begin{array}{@{}c@{}}
                #1
               \end{array} $ \mathstrut}
          (top.south)
          % New version. Line wrapping support.
          node (bottom) [
            anchor=north,
            %inner ysep=.5ex
          ] {
            $ \begin{array}{@{}c@{}}
              #2
            \end{array} $};
    \path[draw,thin,solid] let \p1 = (current bounding box.west),
                               \p2 = (current bounding box.east),
                               \p3 = (top.south)
                           in (\x1,\y3) -- (\x2,\y3);
    % Solid arrow (augmenting the solid line).
    \path[fill] (top.south) ++(2pt,0) -- ++(-4pt,0) -- ++(2pt,-1.5pt) -- cycle;
  \end{tikzpicture}%
}}

%Special version of \reduce with modified baseline, for better rendering of multiline
% LHS and RHS
\newcommand{\reduceCompact}[2]{\hbox{%
  \begin{tikzpicture}[baseline=(bottom), %(top.base), - default, less compact
                      inner xsep=0pt,
                      inner ysep=.3333ex,
                      minimum width=2em]
    \path
          node (top) [inner ysep=0.6ex]{$ \begin{array}{@{}c@{}} #1 \end{array} $ \mathstrut}
          (top.south)
          node (bottom) [anchor=north] {$ \begin{array}{@{}c@{}} #2 \end{array} $};
    \path[draw,thin,solid] let \p1 = (current bounding box.west),
                               \p2 = (current bounding box.east),
                               \p3 = (top.south)
                           in (\x1,\y3) -- (\x2,\y3);
    % Solid arrow (augmenting the solid line).
    \path[fill] (top.south) ++(2pt,0) -- ++(-4pt,0) -- ++(2pt,-1.5pt) -- cycle;
  \end{tikzpicture}%
}}

%\renewcommand{\reduce}[2]{\reduceClassic{#1}{#2}}
\renewcommand{\reduce}[2]{\reduceMulti{#1}{#2}}




%\lstset{captionpos=t,tabsize=3,frame=no,keywordstyle=\color{blue},
%        commentstyle=\color{gray},stringstyle=\color{red},
%        breaklines=true,showstringspaces=false,emph={label},
%        basicstyle=\ttfamily}

% Required in order to make \kall cells inside comments black.
\renewcommand{\kall}[3][black]{\mall{#1}{#2}{#3}}

% Environment "kdefinition" has effect only in poster style, thus in math style may be safely deleted.

%Continuation of a syntax definition on a new line
\newcommand{\syntaxContNewLine}[3][\defSort]{\par\indent\rulebox{%
  $\setlength{\syntaxlength}{\widthof{$\mathrel{::=}$}}%
  \setlength{\syntaxlength}{.5\syntaxlength}%
  \addtolength{\syntaxlength}{\widthof{\syntaxKeyword$#1$}}%
  \hspace{\syntaxlength}%
  \;\;\;\;\;\;\;{#2}$ \ifthenelse{\equal{#3}{}}{}{[#3]}%
  }%\k@markPosition%
}

%Should be put after a syntaxLong.
\newcommand{\syntaxEnd}[3][\defSort]{
  \indent\rulebox{%
  $\setlength{\syntaxlength}{\widthof{$\mathrel{::=}$}}%
  \setlength{\syntaxlength}{.5\syntaxlength}%
  \addtolength{\syntaxlength}{\widthof{\syntaxKeyword$#1$}}%
  \hspace{\syntaxlength}$%
  }%\k@markPosition%
}

\newcommand{\syntaxLong}[3][\defSort]{\rulebox{%
\syntaxKeyword
$
  \begin{array}[t]{@{}l@{}}
  #1 \\
  \mathrel{::=}{#2}
  \end{array}
$ {}%
}%\k@markPosition%
}

% Grigore's idea macro
\newcommand{\idea}[1]{
  \begin{quote}
    \rule{.45\textwidth}{.5pt}\newline
    {\em #1}
    \vspace*{-1ex}\newline \rule{.45\textwidth}{.5pt}
  \end{quote}
}

\newenvironment{ideas}
{ \begin{quote}
    \rule{.45\textwidth}{.5pt}
    \newline
    \begin{em}
} {
    \end{em}
    \vspace*{-1ex}
    \leavevmode
    \newline
    \rule{.45\textwidth}{.5pt}
  \end{quote}
}

%Enforcing black cells
\renewcommand{\kall}[3][white]{\mall{black}{#2}{#3}}
\renewcommand{\kallLarge}[3][white]{\mallLarge{black}{#2}{#3}}
\renewcommand{\kprefix}[3][white]{\mprefix{black}{#2}{#3}}
\renewcommand{\ksuffix}[3][white]{\msuffix{black}{#2}{#3}}
\renewcommand{\kmiddle}[3][white]{\mmiddle{black}{#2}{#3}}

% Settigns required for Chucky's background section
\usepackage{acronym}

\providecommand{\Sec}{}
\renewcommand{\Sec}{Section~}
\newcommand{\Fig}{Figure~}

\newcommand{\cellname}[1]{\textsf{#1}}

\newcommand{\kequation}[2]{\begin{equation*}{\small#2}\end{equation*}}

%Probably a mapsto with spacing
\newcommand{\mapstox}{\small\mathrel{\mapsto}}

%For spacing between cell lines
%\newcommand{\kBR}{\\[0.3em]}

% General
\newcommand{\w}[1]{\ensuremath{\textit{#1}}}
\newcommand{\m}[1]{\ensuremath{\texttt{#1}}}
\newcommand{\s}[1]{\ensuremath{\textsf{#1}}}
\newcommand{\p}[1]{\ensuremath{\left(#1\right)}}
\newcommand{\pl}[1]{\ensuremath{\left\langle#1\right\rangle}}
\newcommand{\OR}{\mbox{ }|\mbox{ }}
\newcommand{\st}{.\mbox{ }}
\newcommand{\finto}{\ensuremath{\stackrel{\mathtt{fin}}{\longrightarrow}}}
\newcommand{\defeq}{\ensuremath{\stackrel{\mathtt{def}}{=}}}
\newcommand{\cond}[1]{\ensuremath{\left\{\begin{array}{ll} #1 \end{array}\right.}}
\newcommand{\lst}[1]{\begin{itemize} {#1} \end{itemize}}
\newcommand{\pby}[1]{\hspace*{\fill}{#1}}
\newcommand{\slide}[2][]{ \begin{frame} \frametitle{#1} {#2} \end{frame} }
\newcommand{\etal}{\textit{et~al.}\xspace}

% For references
\newcommand{\fig}[1]{Figure~\ref{#1}}
\newcommand{\lem}[1]{Lemma~\ref{#1}}
\newcommand{\theo}[1]{Theorem~\ref{#1}}
\newcommand{\coro}[1]{Corollary~\ref{#1}}
\newcommand{\defn}[1]{Definition~\ref{#1}}
\newcommand{\rmrk}[1]{Remark~\ref{#1}}
\newcommand{\exam}[1]{Example~\ref{#1}}
\newcommand{\sect}[1]{$\S$~\ref{#1}}

%\newcommand{\todo}[1]{}
%\newcommand{\todo}[1]{{\textcolor{red}{\textbf{[[{#1}]]}}}}

\usepackage{ucs} % for unicode characters (just for \Rosu and \Serbanuta)
\newcommand{\sh}{\unichar{0537}}
\newcommand{\Sh}{\unichar{0536}}
\PrerenderUnicode{\sh}
\PrerenderUnicode{\Sh}
\newcommand{\Rosu}{Ro{\sh}u\xspace}
\newcommand{\Stefanescu}{{\Sh}tef{\u a}nescu\xspace}

\newcommand{\JS}{JavaScript\xspace}
\newcommand{\ES}{ECMAScript\xspace}
%\newcommand{\K}{\ensuremath{\mathbb{K}}\xspace}
%\newcommand{\KJS}{\ensuremath{\mathbb{K}}JS\xspace}
\newcommand{\KJS}{KJS\xspace}
\newcommand{\LJS}{\ensuremath{\lambda_{\w{JS}}}\xspace}
\newcommand{\spec}{specification\xspace}

\lstdefinelanguage{JavaScript}{
  keywords={break, case, catch, continue, debugger, default, delete, do, else, finally, for, function, if, in, instanceof, new, return, switch, this, throw, try, typeof, var, void, while, with},
  morecomment=[l]{//},
  morecomment=[s]{/*}{*/},
  morestring=[b]',
  morestring=[b]",
  sensitive=true
}

\definecolor{orange}{rgb}{1,0.5,0}
\definecolor{darkgreen}{rgb}{0.0, 0.5, 0.0}
\newcommand{\note}[2]{\textbf{\textit{\textcolor{#1}{[[{#2}]]}}}}
\newcommand{\marker}[1]{} %{\note{orange}{{#1}}}
\newcommand{\daejun}[1]{\note{red}{Daejun: {#1}}}
\newcommand{\andrei}[1]{\note{darkgreen}{Andrei: {#1}}}
\newcommand{\grigore}[1]{\note{blue}{Grigore: {#1}}}



\definecolor{codegreen}{rgb}{0,0.6,0}
\definecolor{codegray}{rgb}{0.5,0.5,0.5}
\definecolor{codepurple}{rgb}{0.58,0,0.82}
\definecolor{backcolour}{rgb}{0.95,0.95,0.92}


\lstdefinestyle{Bash}{
    language=Bash,                % choose the language of the code
    basicstyle=\footnotesize,       % the size of the fonts that are used for the code
    numbers=left,                   % where to put the line-numbers
    numberstyle=\tiny\color{codegray},      % the size of the fonts that are used for the line-numbers
    stepnumber=1,                   % the step between two line-numbers. If it is 1 each line will be numbered
    numbersep=5pt,                  % how far the line-numbers are from the code
    backgroundcolor=\color{white},  % choose the background color. You must add \usepackage{color}
    showspaces=false,               % show spaces adding particular underscores
    showstringspaces=false,         % underline spaces within strings
    showtabs=false,                 % show tabs within strings adding particular underscores
    frame=single,           % adds a frame around the code
    %tabsize=2,          % sets default tabsize to 2 spaces
    captionpos=b,           % sets the caption-position to bottom
    breaklines=true,        % sets automatic line breaking
    breakatwhitespace=false,    % sets if automatic breaks should only happen at whitespace
    escapeinside={\%*}{*)},          % if you want to add a comment within your code
    commentstyle=\color{gray},
    keywordstyle=\color{blue},
    morekeywords={andnq, jp, jz, movw, movq, xorq, orq, retq, pushw}
}

\lstdefinestyle{C++}{
    language=C++,                % choose the language of the code
    basicstyle=\footnotesize,       % the size of the fonts that are used for the code
    numbers=left,                   % where to put the line-numbers
    numberstyle=\tiny\color{codegray},      % the size of the fonts that are used for the line-numbers
    stepnumber=1,                   % the step between two line-numbers. If it is 1 each line will be numbered
    numbersep=5pt,                  % how far the line-numbers are from the code
    backgroundcolor=\color{white},  % choose the background color. You must add \usepackage{color}
    showspaces=false,               % show spaces adding particular underscores
    showstringspaces=false,         % underline spaces within strings
    showtabs=false,                 % show tabs within strings adding particular underscores
    frame=single,           % adds a frame around the code
    %tabsize=2,          % sets default tabsize to 2 spaces
    captionpos=b,           % sets the caption-position to bottom
    breaklines=true,        % sets automatic line breaking
    breakatwhitespace=false,    % sets if automatic breaks should only happen at whitespace
    escapeinside={\%*}{*)},          % if you want to add a comment within your code
    commentstyle=\color{gray},
    keywordstyle=\color{blue},
}

\lstdefinestyle{SMTLIB}{
    language=Java,
    basicstyle=\footnotesize,       % the size of the fonts that are used for the code
    numbers=left,                   % where to put the line-numbers
    numberstyle=\tiny\color{codegray},      % the size of the fonts that are used for the line-numbers
    stepnumber=1,                   % the step between two line-numbers. If it is 1 each line will be numbered
    numbersep=5pt,                  % how far the line-numbers are from the code
    backgroundcolor=\color{white},  % choose the background color. You must add \usepackage{color}
    showspaces=false,               % show spaces adding particular underscores
    showstringspaces=false,         % underline spaces within strings
    showtabs=false,                 % show tabs within strings adding particular underscores
    frame=single,           % adds a frame around the code
    %tabsize=2,          % sets default tabsize to 2 spaces
    captionpos=b,           % sets the caption-position to bottom
    breaklines=true,        % sets automatic line breaking
    breakatwhitespace=false,    % sets if automatic breaks should only happen at whitespace
    escapeinside={(*}{*)},          % if you want to add a comment within your code
    commentstyle=\color{gray},
    keywordstyle=\color{blue},
    morekeywords={bvand, bvnot, concat, extract, bvxor}
}

\lstdefinestyle{KRULE}{
    %language=Java,
    %basicstyle=\footnotesize,      
    basicstyle=\scriptsize,
    backgroundcolor=\color{white},  % choose the background color. You must add \usepackage{color}
    showspaces=false,               % show spaces adding particular underscores
    showstringspaces=false,         % underline spaces within strings
    showtabs=false,                 % show tabs within strings adding particular underscores
    frame=single,           % adds a frame around the code
    %tabsize=2,          % sets default tabsize to 2 spaces
    captionpos=b,           % sets the caption-position to bottom
    breaklines=true,        % sets automatic line breaking
    breakatwhitespace=false,    % sets if automatic breaks should only happen at whitespace
    escapeinside={(*}{*)},          % if you want to add a comment within your code
    commentstyle=\color{gray},
    morecomment=[l]{//},
    keywordstyle=\color{blue},
    %morekeywords={regstate, stackmem, andBool, requires, ensures, codemem, memstate, and}
    morekeywords={andBool, requires, ensures, and, rule}
}

\lstdefinestyle{KRULEWOBORDER}{
    %language=Java,
    %basicstyle=\footnotesize,      
    basicstyle=\scriptsize,
    backgroundcolor=\color{white},  % choose the background color. You must add \usepackage{color}
    showspaces=false,               % show spaces adding particular underscores
    showstringspaces=false,         % underline spaces within strings
    showtabs=false,                 % show tabs within strings adding particular underscores
    %frame=single,           % adds a frame around the code
    %tabsize=2,          % sets default tabsize to 2 spaces
    captionpos=b,           % sets the caption-position to bottom
    breaklines=true,        % sets automatic line breaking
    breakatwhitespace=false,    % sets if automatic breaks should only happen at whitespace
    escapeinside={(*}{*)},          % if you want to add a comment within your code
    commentstyle=\color{gray},
    morecomment=[l]{//},
    keywordstyle=\color{blue},
    morekeywords={regstate, stackmem, andBool, requires, ensures, codemem, memstate, and}
}

\lstdefinestyle{SIMPRULES}{
    language=Java,
    basicstyle=\footnotesize,       % the size of the fonts that are used for the code
    backgroundcolor=\color{white},  % choose the background color. You must add \usepackage{color}
    escapeinside={(*}{*)},          % if you want to add a comment within your code
    commentstyle=\color{gray},
    morecomment=[l]{//},
}


%
% end of the preamble, start of the body of the document source.
\begin{document}

%
% The "title" command has an optional parameter, allowing the author to define a "short title" to be used in page headers.
\title[\ISA Semantics]{A Complete Formal Semantics of \ISA User-Level Instruction Set Architecture}

%
% The "author" command and its associated commands are used to define the authors and their affiliations.
% Of note is the shared affiliation of the first two authors, and the "authornote" and "authornotemark" commands
% used to denote shared contribution to the research.
%\author{Ben Trovato}
%\authornote{Both authors contributed equally to this research.}
%\email{trovato@corporation.com}
%\orcid{1234-5678-9012}
%\author{G.K.M. Tobin}
%\authornotemark[1]
%\email{webmaster@marysville-ohio.com}
%\affiliation{%
%  \institution{Institute for Clarity in Documentation}
%  \streetaddress{P.O. Box 1212}
%  \city{Dublin}
%  \state{Ohio}
%  \postcode{43017-6221}
%}

\author{Sandeep Dasgupta}
\affiliation{%
  \institution{University of Illinois at Urbana Champaign, USA}}
%  \streetaddress{1 Th{\o}rv{\"a}ld Circle}
%  \city{Hekla}
%  \country{USA}}
\email{sdasgup3@illinois.edu}

\author{Daejun Park}
\affiliation{%
  \institution{University of Illinois at Urbana Champaign, USA}}
\email{dpark69@illinois.edu}

\author{Theodoros Kasampalis}
\affiliation{%
  \institution{University of Illinois at Urbana Champaign, USA}}
\email{kasampa2@illinois.edu}

\author{Vikram S. Adve}
\affiliation{%
  \institution{University of Illinois at Urbana Champaign, USA}}
\email{vadve@illinois.edu}

\author{Grigore Ro\c{s}u}
\affiliation{%
  \institution{University of Illinois at Urbana Champaign, USA}}
\email{grosu@illinois.edu}

%\author{Valerie B\'eranger}
%\affiliation{%
%  \institution{Inria Paris-Rocquencourt}
%  \city{Rocquencourt}
%  \country{France}
%}
%
%\author{Aparna Patel}
%\affiliation{%
% \institution{Rajiv Gandhi University}
% \streetaddress{Rono-Hills}
% \city{Doimukh}
% \state{Arunachal Pradesh}
% \country{India}}
% 
%\author{Huifen Chan}
%\affiliation{%
%  \institution{Tsinghua University}
%  \streetaddress{30 Shuangqing Rd}
%  \city{Haidian Qu}
%  \state{Beijing Shi}
%  \country{China}}
%
%\author{Charles Palmer}
%\affiliation{%
%  \institution{Palmer Research Laboratories}
%  \streetaddress{8600 Datapoint Drive}
%  \city{San Antonio}
%  \state{Texas}
%  \postcode{78229}}
%\email{cpalmer@prl.com}
%
%\author{John Smith}
%\affiliation{\institution{The Th{\o}rv{\"a}ld Group}}
%\email{jsmith@affiliation.org}
%
%\author{Julius P. Kumquat}
%\affiliation{\institution{The Kumquat Consortium}}
%\email{jpkumquat@consortium.net}

%
% By default, the full list of authors will be used in the page headers. Often, this list is too long, and will overlap
% other information printed in the page headers. This command allows the author to define a more concise list
% of authors' names for this purpose.
\renewcommand{\shortauthors}{S. Dasgupta et al.}

%% Abstract
%% Note: \begin{abstract}...\end{abstract} environment must come
%% before \maketitle command
\begin{abstract}
% We present a complete, thoroughly tested, and executable formal semantics of \ISA user-level instruction set architecture in \K framework.
% Being executable, the model has been tested rigorously, using a test-suite of 7,000+ test inputs and \ISA programs corresponding to GCC torture tests, against actual hardware leading to revealing bugs in Intel Manual and related projects.
% Using \K framework's formal reasoning tools (such as deductive program verifier, symbolic execution, and program equivalence checker), we demonstrated the applicability of our semantics in formal analysis and verification of non-trivial machine programs and proving equivalence of \ISA programs across optimizations.
% Also, the model has been applied to generate test inputs to trigger a known security vulnerability.
We present the most complete and thoroughly tested formal semantics of \ISA to date.
Our semantics faithfully formalizes all the non-deprecated, sequential user-level instructions of the \ISA Haswell instruction set architecture.
This totals \currentIS{} instruction variants, corresponding to \currentIntel{} mnemonics. % ~\cite{IntelManual} (Section~\ref{sec:IC}).
The semantics is fully executable and has been tested against more than 7,000 instruction-level test cases and the GCC torture test suite. % using the co-simulation method,
This extensive testing paid off, revealing bugs in both the \ISA reference manual and other existing semantics.
\AEC{We also illustrate potential applications of our semantics in different formal analyses such as symbolic execution, deductive verification, and translation validation, and discuss how it can be useful for processor verification.}
% verifying the functional correctness of a non-trivial \ISA binary, translation validation of binary optimizations, and generating inputs to trigger a known security vulnerability.
\end{abstract}


%
% The code below is generated by the tool at http://dl.acm.org/ccs.cfm.
% Please copy and paste the code instead of the example below.
%
\begin{CCSXML}
<ccs2012>
<concept>
<concept_id>10003752.10003766.10003767.10003769</concept_id>
<concept_desc>Theory of computation~Rewrite systems</concept_desc>
<concept_significance>500</concept_significance>
</concept>
<concept>
<concept_id>10003752.10003790.10002990</concept_id>
<concept_desc>Theory of computation~Logic and verification</concept_desc>
<concept_significance>500</concept_significance>
</concept>
<concept>
<concept_id>10003752.10010124.10010131.10010134</concept_id>
<concept_desc>Theory of computation~Operational semantics</concept_desc>
<concept_significance>500</concept_significance>
</concept>
<concept>
<concept_id>10003752.10010124.10010138</concept_id>
<concept_desc>Theory of computation~Program reasoning</concept_desc>
<concept_significance>500</concept_significance>
</concept>
<concept>
<concept_id>10010583.10010717.10010721.10010725</concept_id>
<concept_desc>Hardware~Simulation and emulation</concept_desc>
<concept_significance>500</concept_significance>
</concept>
<concept>
<concept_id>10010583.10010717.10010721.10010727</concept_id>
<concept_desc>Hardware~Theorem proving and SAT solving</concept_desc>
<concept_significance>500</concept_significance>
</concept>
<concept>
<concept_id>10011007.10011006.10011039</concept_id>
<concept_desc>Software and its engineering~Formal language definitions</concept_desc>
<concept_significance>500</concept_significance>
</concept>
</ccs2012>
\end{CCSXML}

\ccsdesc[500]{Theory of computation~Rewrite systems}
\ccsdesc[500]{Theory of computation~Logic and verification}
\ccsdesc[500]{Theory of computation~Operational semantics}
\ccsdesc[500]{Theory of computation~Program reasoning}
\ccsdesc[500]{Hardware~Simulation and emulation}
\ccsdesc[500]{Hardware~Theorem proving and SAT solving}
\ccsdesc[500]{Software and its engineering~Formal language definitions}

%
% Keywords. The author(s) should pick words that accurately describe the work being
% presented. Separate the keywords with commas.
\keywords{\ISA, ISA specification, Formal Semantics}

%
% This command processes the author and affiliation and title information and builds
% the first part of the formatted document.
\maketitle

    \section{Introduction}
\label{sec:Intro}

%% Why X86 is important?
% \ISA is undoubtedly the most widely used instruction set architecture on
% servers and personal computers which have grown to a remarkable complexity over
% the past few decades. 
%% Why analysis on X86 is important?
% Because of its wide-spread use, tools related to analysis and reasoning on the
% binary code are pervasive in software engineering and security research
% ~\cite{X}. Even there are situations where binary analysis seems more desirable than that
% on the source code. Examples are Commercial Off-The-Shelf software,
%   legacy code, or malware where the source code is not available. Even when
%   the source code is available, we cannot trust it when the compiler
%   generating the binary is not in the trusted computing base~\cite{Thompson}.
%   Lastly, even a trusted compiler can produce code which is semantically
%   different from the binary~\cite{WYSINWYE} and hence the desire to analyze the
%   binary itself.

The \ISA instruction set architecture (ISA) is one of the most complex and widely used ISAs on servers and desktops, 
and ensuring the correctness of the \ISA binary code is important.
%
The ability to directly reason about the binary code is desirable, not only because it allows to analyze the binary even when the source code is not available (e.g., legacy code or malware), but also because it avoids the need to trust the correctness of compilers~\cite{Thompson,WYSINWYE}.

% %
% Indeed, there exist various binary analysis tools, including those for software emulation and virtualization~\cite{QEMU:USENIX05,Valgrind:ENTCS03,DynamoRIO:2004,Pin:2005},
% malware analysis~\cite{BitBlaze:2008,BAP:CAV11,Egele:USENIX07,Yin:CCS07},
% reverse engineering~\cite{McSema:Recon14,Angr,Radare2}
% and sand-boxing~\cite{Kiriansky:2002:SEV,Erlingsson:2006,Yee:2009}.
% %
% These tools, however, are not designed to formally reason about the binary 

% depend, either explicitly or implicitly, on
% correct modeling of
%  the semantics of x86-64 instructions. 

A formal semantics of \ISA is required for formal reasoning about binary code, one of the strongest ways to ensure its correctness.
%
An \emph{executable} semantics is especially powerful because it allows direct testing to gain confidence in the definitions of the semantics, and also because it can allow additional tools based on symbolic execution, like deductive verification and symbolic test generation.
%
Completely formalizing the semantics of \ISA, however, is challenging especially due to the complexity and the sheer number of instructions that are informally specified in approximately 3,000-page standard~\cite{IntelManual}.

\paragraph{Existing Semantics for \ISA:}
%
To date, to the best of our knowledge, despite several explicit attempts~\cite{Heule2016a,Goel:FMCAD14,Goel:ProCoS17,Goel:VTTE13:ACP:2958657.2958669} and other related systems~\cite{Leroy:2009,Remill,TSL:TOPLAS13,Hasabnis:ASPLOS16,Hasabnis:FSE16},
no \emph{complete} formal semantics of \ISA exists that can be used for formal reasoning about x86 binary programs.
%
Heule~\etal \cite{Heule2016a} presented a formal semantics of \ISA, but it covers only a fragment ($\sim$\strataPerc{}) of all instructions; as the authors of \cite{Heule2016a} candidly admitted, their synthesis methodology proved insufficient to add the remaining instructions primarily due to limitations of the underlying synthesis engine. 
%
Moreover, their semantics misses certain essential details (Section~\ref{sec:Approach} \& \ref{sec:Eval}).
% and it has not been demonstrated how to use their semantics to formally reason about the functional correctness of the \ISA binary.\footnote{Although they provide a tool that can extract SMT formulae describing each instruction's behavior from their semantics, we found errors in their tool and thus the generated SMT formulae. Also, most of the SMT solvers are not designed to support rich reasoning principles.}
% natively support all the reasoning principles of the full-fledged proof assistants.}
% for the general purpose formal reasoning.}
%
% it is not clear how to lift their semantics to the full-fledged theorem prover to be able to reason about the full functional correctness of the \ISA binary.
% Although it can provide the semantics in the form of the SMT formulae, the SMT solver 
%
Goel~\etal~\cite{Goel:FMCAD14,Goel:ProCoS17,Goel:VTTE13:ACP:2958657.2958669}, on the other hand, specified a formal semantics in the \SC{ACL2} proof assistant~\cite{ACL2:Kaufmann2000}, allowing to reason about functional correctness, but their semantics covers only a small fragment ($\sim$\goelPerc{}) of all user-level instructions.
%\Comment{Section 2 says they had only 191 unique opcodes, which is only about 20\%, and even less if you include system instructions. Which is correct?}
%
There also have been several attempts~\cite{Angr1,BAP:CAV11,Radare2,Hasabnis:FSE16} to \emph{indirectly} describe the \ISA semantics, where they define an intermediate language (IL), specify the IL semantics, and translate \ISA to the IL.
This indirect semantics, however, may not be general enough to be used for different types of formal analyses, since the IL might be designed with specific purposes in mind, not to mention that the translation may miss certain important details of the instruction behaviors.
%
Refer to Section~\ref{sec:RW} for a more detailed comparison to existing semantics.

\paragraph{Our Approach:}
%
We present the most complete and thoroughly tested formal semantics of user-level \SC{\ISA assembly instructions}\footnote{\SC{The current work do not include a formal model of the binary instruction decoder. Note that, all future references of \ISA{} ``program(s)'' or ``instructions(s)'', in the context of our model, are meant to refer to the ``assembly language programs(s)'' or ``assembly instruction(s)''.}} to date.
We employed the \K framework~\cite{k-primer-2013-v32} (Section ~\ref{sec:KF}) as our formalism medium to leverage its capability of deriving various correct-by-construction formal analysis tools directly from the language semantics.
We took Heule~\etal~\cite{Heule2016a}'s semantics (Section ~\ref{sec:prelimstrata}) as our starting point to avoid duplicating the formalization effort. % made by the formal semantics research community.
We made several corrections or improvements to this semantics, to improve both soundness and efficiency.
We \emph{automatically} translated their semantics into \K, and cross-checked the translated semantics against the original using an SMT solver.
% cross-checked it by comparing the SMT formulae generated by each formalism, increasing our convince of the faithfulness of the translation.
We \emph{manually} specified the semantics of the remaining instructions faithfully consulting the Intel manual~\cite{IntelManual} to obtain the complete semantics. A manual specification \SC{may sound like a daunting effort} at first, but the fact that (1) \ISA is largely stable and changes slowly over time, and (2) the state-of-the-art synthesis techniques for language semantics (notably, \Strata~\cite{Heule2016a} and Hasabnis \etal~\cite{Hasabnis:ASPLOS16, Hasabnis:FSE16}) suffer from scalability and/or faithfulness issues (see Section~\ref{sec:Approach:Overview} and \ref{sec:RW} for details), makes the effort worth undertaking.


% \Added{Moreover, an important message of our paper is that complete formal semantics of x86 is possible, and that is not only useful in itself but also to generate formal analysis tools.}



%($\sim$40\%), obtaining the complete semantics.
Like closely related previous work~\cite{Goel:FMCAD14,Heule2016a}, we omit the relaxed memory model of \ISA and thus the concurrency-related operations.
Modelling concurrency is a complex but relatively orthogonal problem in the presence of small-step operational semantics, as shown in prior work~\cite{Sarkar:POPL09,Owens:x86-TSO}, where they have integrated their memory model with a small subset of $32$-bit x86 instruction set.
We believe that integrating such a memory model into our instruction semantics is a promising direction toward rigorously reasoning about real-world programs running on modern multiprocessors. We leave it for future work.


\paragraph{Contributions:}
%
We describe our primary contributions:

%DSAND: Include Intel counts
\begin{itemize}
\item
\emph{Completeness.}
We present the most complete formal semantics of \ISA to date.
\SC{Specifically, our semantics formalizes all the user-level instructions of the \ISA Haswell ISA (that is,
\currentIS{} instructions covering \currentIntel{} mnemonics~\cite{IntelManual}), except deprecated ones (\Xmmx{} instructions),
the AES cryptography extensions (\crypto{} instructions), and the system \& concurrency-specific instructions (\system{} instructions) (Refer Section~\ref{sec:IC})}.
\item
\emph{Faithfulness.}
Being executable, the semantics of \SC{\emph{each}} instruction has been thoroughly tested against 7,000+ test cases using the co-simulation method (Section~\ref{sec:Eval}).
We found errors in both the \ISA standard document and other existing semantics including the baseline semantics (Section~\ref{sec:Eval}). \SC{Note that, the testing of floating point instructions is facilitated by the fact K already has matured library support for floating point theories which we augmented to support modeling such instructions. In Section~\ref{sec:limit}, we reported a precision issue with our floating point library support.}
\item
\emph{Usability \& portability.}
We demonstrate that our semantics can be used for various formal analyses, such as symbolic execution, deductive program verification, and program equivalence checking (Section~\ref{sec:Appl}).
The \K framework also enables to represent our semantics as SMT theories,
% that can be handled by various SMT solvers,
which allows others to easily reuse our semantics for their own purposes.
\item
\emph{Semantics development practice.}
Reflecting our x86 semantics development effort, we identify certain important aspects to be considered and/or ensured when specifying a large instruction set architecture semantics, which we believe can be also applied to other kinds of large language semantics to certain extent (Section~\ref{sec:lesson-learned}).
\end{itemize}
%
% \paragraph{Artifacts}
Our formal semantics is publicly available at~\cite{Suppl}.


% In this paper, we present the most complete and thoroughly tested formal semantics of \ISA to date.
% Specifically, our semantics faithfully formalizes all the user-level instructions (\currentIS{} in total) of the \ISA Haswell ISA~\cite{IntelManual}.
% Our semantics is specified in the K framework~\cite{k-primer-2013-v32} that allows us to execute our semantics, and derive various correct-by-construction formal analysis tools directly from the semantics.
% Being executable, our semantics has been thoroughly tested against 7,000+ test cases using the co-simulation method (Section~\ref{sec:Eval}).
% We also demonstrate that our semantics can be used for various formal analyses, such as symbolic execution, deductive program verification, and program equivalence checking (Section~\ref{sec:Appl}).
% The K framework also allows us to represent our semantics in the SMT theories as well,
% % that can be handled by various SMT solvers,
% which allows others to easily reuse our semantics for their own purposes.




\subsection{Challenges in Formalizing \ISA}
\label{sec:challenges}

% In addition to the sheer number of instructions to be specified,
% % and the ambiguity of the informal standard document to consult,
% the following aspects make it challenging to completely specify the formal semantics of \ISA.

\paragraph{Size and Complexity:}
%
\SC{The \ISA ISA has a vast number of instructions, partly because of a large number of complex instructions and partly because it keeps most of the legacy and deprecated instructions ($\sim$ \Xmmx{}+) for the sake of backwards compatibility.}
It consists of \totalIntel{} mnemonics, and each mnemonic admits several variants, depending on the types (i.e., register, memory, or constant) and the size (i.e., the bit-width) of operands.

\paragraph{Inconsistent Instruction Variants:}
%
Some variants have quite divergent behaviors more than the difference of their type and size. For example, \instr{movsd}, one of the 128-bit SSE instructions, has very different behaviors depending on whether the type of the source operand is register or memory; it clears the higher 64 bits of the target register only when the source type is memory.
Indeed, we revealed bugs in other semantics due to their incorrect generalization of the variants' behavior (Details in Section ~\ref{sec:Approach}, Instruction Variants).

\paragraph{Ambiguous Documentation:}
%
The \ISA reference manual informally explains the instruction behaviors, leaving certain details unspecified or ambiguous, which required us to consult with an actual processor implementation to clarify such details.
%
Completely formalizing the vast number of instructions with carefully identifying all the corner cases from the informal document, thus, is highly non-trivial.
% is a huge effort that may not seem to be feasible.
% , simply because of the sheer number of instructions to be formalized.



% \ISA has the special register \reg{rflags} that stores the current state of the processor.



\paragraph{Undefined Behaviors:}
%
The \ISA standard also admits undefined behaviors that are implementation-dependent.
Many instructions (\undefIntel{}\footnote{\label{note1}These numbers are obtained by parsing the official manual ``Volume 2: Instruction Set Reference'' and cross checked with projects~\cite{Stoke2013, Felix} investing similar efforts.} out of \totalIntel{}) have undefined behaviors: their output values of the destination register or the \reg{rflags} register are undefined in certain cases.
% behaviors, for which each processor can choose any behavior.
% For example, the bit-scan-forward instruction \instr{bsf} that computes the bit index of the least significant set bit in the source operand is not defined when the source operand's value is 0.
That is, the processor is free to choose any behavior in the undefined case.
% (i.e., no bit 1 appears), however, its output is implementation-dependent.
% More than \undefPerc{}\% of all instructions admit undefined behaviors!
%

Many existing semantics, however, simply ``define'' the undefined behaviors by
% consulting with an actual processor implementation.
following a specific behavior taken by a processor implementation.
This approach is problematic because they do not capture all possible behaviors of different processor implementations.
Indeed, we found discrepancies between existing semantics in specifying the undefined behaviors, where different semantics are valid only for different groups of processors.
That is, such semantics are not adequate to formally reason about universal properties (e.g., portability) of a program that need to be satisfied for all standard-conforming processors.
%
For example, the parity flag \reg{pf} is undefined in the logical-and-not instruction \instr{andn}, where the processor implementation is allowed to either update the flag value (to 0 or 1), or keep the previous value.
We found, e.g., that Remill~\cite{Remill} updates the flag with 0, whereas Radare~\cite{Radare2} keeps it unmodified.
% We found that Strata~\cite{Heule2016a} updates the flag based on the result of the \instr{andn} operation, whereas a binary analysis tool Radare~\cite{Radare2}, for example, keeps it unmodified.
%
Identifying and faithfully specifying all of the undefined behaviors, thus, are desirable but challenging.

\SC{In our semantics, we faithfully modeled the undefined value as a unique symbol (called \s{undef}) whose value is nondeterministically decided each time within the proper range.}
These nondeterministic values are enough to capture and formally reason about all possible behaviors of the instructions for different processors (and even any future, standard-conforming processor).
% Regarding the instruction-level testing (Section~\ref{sec:Eval}), we consider the \s{undef} symbol to be matched with any concrete value provided by the hardware, so that we can test the instructions modulo the undefined behaviors.




% \paragraph{Undefined, Implementation-Dependent Behaviors}

% According to the \ISA standard, 

% We found that other semantics do not faithfully model the undefined behaviors, simply following a specific behavior taken by a processor implementation.
%









% many of the existing semantics do not faithfully specify the implementation-dependent semantics, and divergent behaviors across the different semantics.

% Naively specifying the implementation-dependent behaviors by following the behaviors of an existing processor is problematic, because it cannot capture all possible behaviors of different processors.\footnote{Even if all of the existing processors agree on a certain behavior, it may not be the case in the future processors.}
% That is, a program verified w.r.t. such a naive semantics may have different behaviors in another processor (in the future).
% Identifying and faithfully specifying all of the implementation-dependent behaviors are challenging.
% Indeed, we found that many of the existing semantics do not faithfully specify the implementation-dependent semantics, and divergent behaviors across the different semantics. % , which may not be sound for different processors.

% A naive approach of specifying the implementation-dependent behaviors would be to follow the behaviors of an existing processor.
% This approach has the benefits of having the co-simulation based testing straightforward.
% However, the naively defined semantics cannot capture all possible behaviors of different processors.\footnote{Even if all of the existing processors agree on a certain behavior, it may not be the case in the future processors.}
% That is, a program verified w.r.t. such a naive semantics may have different behaviors in another processor (in the future).


% In effect, these
% approaches restricts the value of the flag to a concrete value and hence a
% semantics model based on these approaches prevent exploration of paths feasible
% in some processor implementation.  Therefore, it is desirable to identify these
% cases where a register or a flag could be undefined and to encode this
% information in the model in such a way so as to assist exploration of all
% resulting execution paths.






%  However, such a modeling is difficult to obtain given the fact that the ISA is
%  overly complex and the only published description of the \ISA ISA is the Intel
%  manual~\cite{IntelManual} with over 3000 pages written in an ad-hoc combination of
%  English and pseudo-code. Also Intel does not appear to have a formal model (not even internally) that fully defines CPU behavior (~\cite{Amit:SOSP15}, Section 4.1). This informal nature of the reference specification
%  imposes a challenge in ensuring correctness of the developed formal 
%  semantics.
%  Most of the existing binary analysis tools mitigate the challenge by manually modeling  the specification of the semantics in their analysis IR which are then validated against the actual hardware to attain faithfulness of the model.

% %% What are we offering?
% Our goal is to formally model the semantics of all the user-level \ISA Haswell ISA, which is used widely for server and desktop machines nowadays. This work will not only help in formal reasoning on binary programs but also serve as a comparison reference for the existing binary analysis projects. In this paper we mention the challenges we faced and the lessons we learned while doing so. 

% \subsection{Why Yet Another \ISA Semantics}
% A formal semantics of a language is foundational to any formal reasoning about programs written in that language and serves as a trust base on which the faithfulness of downstream analyses and reasoning tools depends. Following are some of the desirable properties of a formal model:
% \begin{itemize}
%     \item \textbf{Completeness: }To make the model applicable to real world programs.
%     \item \textbf{Executable: }To compare the model against a reference implementation, which in turn enhances the faithfulness of the model.     
%     \item \textbf{Applicable for program reasoning \& verification: }To reason about arbitrary high-level properties on the program using a general reasoning infrastructure such as theorem prover.
%     \item \textbf{Same artifact being used for both execution and formal reasoning: } To make the trust base smaller by having a single artifact used for both execution and formal reasoning. This also adds to the faithfulness of the model. Also having separate formalizations incur the overhead of maintaining both.
%     \revisit{an example to show the relevance of this point}
% \end{itemize}

% %\revisit{Can we say McSema has formal spec? as I am not sure if McSema llvm based spec is to be called formal}
% Several research groups have implemented, with considerable effort, their own
%  instruction semantics specification for the \ISA instruction set. Notables are \Strata~\cite{Heule2016a} and Goel et al.~\cite{Goel:FMCAD14, Goel:ProCoS17, Goel:VTTE13:ACP:2958657.2958669} which are considered state of the art in defining the formal semantics. 
%  \Strata provides fairly complete support of instruction semantics but we found that the semantics specifications are not ready to be used directly for a solid foundation of formal reasoning. Indeed, we found several issues and inconsistencies in their semantics (more details in section~\ref{sec:Eval}).  Also, it is not clear how to connect the \Strata semantics to a full-fledged theorem prover in order to reason about the full functional correctness of the \ISA binary programs. On the other hand, the Goel et al. semantics is far from being complete w.r.t user-level instruction coverage ($25.54\%$). As shown in Table~\ref{table:RW}, no existing semantics meet all the desired properties mentioned above (details about the comparison are mentioned in section~\ref{sec:RW}). Hence we developed a formal \ISA semantics
% in order to have a single, clean-slate semantics that can be used not
% only as a reference model for \ISA ISA, but also to develop formal
% analysis tools for it.






% \subsection{Our Appraoch}\label{sec:M} 
% %% What is the methodology
% Because of the huge volume of the user-level instructions available in Intel
% Manual, we avoided modeling the entire set from scratch and decided to reuse
% the effort and research already invested by other projects. Towards that goal,
%     we chose to borrow the instruction semantics from project
%     \Strata~\cite{Heule2016a}, which supports \strataPerc{} of the total user-level
%     instructions. We modeled the semantics of remaining \currentManualPerc{} instructions by
%     carefully consulting the Intel manual and put significant effort in
%     validating the semantics (as detailed in section~\ref{sec:Eval}). More details about this
%     aspect of our approach is presented in section~\ref{sec:harvestsema}. The
%     choice of \Strata is based on the fact that 1. It has fairly complete
%     instruction support, and 2. The semantic specifications are better suited,
%     to be used as a starting point  towards defining formal semantics, than
%     other specifications used in related projects having larger instruction
%     support (refer to section~\ref{sec:RW} for more details). 

% As detailed in section~\ref{sec:x86sema}, we employed \K~\cite{k-primer-2013-v32} to define the formal semantics of \ISA
% ISA. Given that semantics, \K provides, at no additional cost,  an execution
% engine, which yields an interpreter for the defined language, as well as a
% sound and relatively complete deductive verification system based on symbolic
% execution, which can be used to reason about \ISA programs.  \cmt{With that our
%   x86 model serves both as an executable instruction-set simulator and a formal
%     specification that is used to proof high-level properties about machine
%     code.} The choice of \K is based on the additional fact that it is highly
%     specialized in defining language semantics (as demonstrated by the full
%         formal semantics of production languages like C and
%         LLVM~\cite{Ellison,KC, KLLVM}) and  has high level of automation and
%     expressive power\Qd{should I add "than other contemporary language
%       formalization framework". Also  let me know if there is a better way to
%         put the last sentence}.  


 
% %%%%%%%%%%%%%%%%%%%%%%%%%%%%%%%%%%%%%%%%%%%%%%%%%%%%%%%%%%%%%%%%%%%%%%
% %%%%%%%%%%%%%%%%%%%%%%%%%%%%%%%%%%%%%%%%%%%%%%%%%%%%%%%%%%%%%%%%%%%%%% 
% \subsection{Contribution}
% Our contributions can be listed as follows:
% \begin{itemize}
%     \item Developed a complete formal semantics of the \ISA user-level ISA. 
%     \item Thoroughly tested the instructions semantics using
%     \begin{itemize}
%         \item Instruction level testing
%         \item Program level testing
%     \end{itemize}
%     While doing so, we identify several important errors in pre-existing formalizations including Intel Manuals. Moreover, our specification of the x86 ISA is regularly validated to increase faithfulness of the semantics and the trust in the applicability of the results of formal analysis.
%     \item Applied the semantics to formal reasoning on \ISA programs.
%     \begin{itemize}
%         \item Verified functional correctness of toy \Qd{I think non-trivial is a better choice than toy} programs, like \emph{sum-to-n}.
%         \item Generated test cases using symbolic execution, for detecting security vulnerability.
%         \item Checked equivalences between \ISA programs across different optimizations.
%         \item Verified the translation of binary lifters targeting LLVM IR~\cite{McSema:Recon14, FCD}.
%         \revisit{Need to reorder the sequence  once we have the x86 -> llvm validator in place} 
%     \end{itemize}
% \end{itemize}

% \subsection{Outline}
% Next, we present our formal, executable model of the x86 ISA from an
% engineering standpoint and describe our design decisions and challenges in
% extending Strata in detail.

\section{Preliminaries}
\label{sec:Prelim}

Here we provide background on the \K framework and the \Strata project~\cite{Heule2016a} (used for our baseline semantics).

% briefly explain pieces of \ISA ISA necessary for our
% presentation. We also talk about \K, a semantics engineering tool which we
% chose to formalize our semantics into and  \Strata which gives us a head-start towards modeling the semantics of \ISA instructions.

%%% (VSA) DISABLE X86 BACKGROUND: IT IS WIDELY KNOWN
%\subsection{\ISA Instruction Set Architecture}

\ISA is the 64-bit extension of x86, a family of backward-compatible ISAs.
% x64 is a generic name for the 64-bit extensions to Intel's and AMD's 32-bit x86 instruction set architecture (ISA). 
%
We briefly explain some details of the architecture.

% Registers:
\ISA has the sixteen 64-bit general purpose registers (\reg{rax}--\reg{rdx}, \reg{rsi}, \reg{rdi}, \reg{rsp}, \reg{rbp}, \reg{r8}--\reg{r15}), and the two $64$-bit special registers (\reg{rip} and \reg{rflags}).
The lower $32$-, $16$- and $8$-bit portions of the register are referenced by the sub-register variants, e.g., \reg{eax}, \reg{ax}, and \reg{al} for \reg{rax}, respectively.
% \cmt{All these registers are used for storing integer values.}
The Haswell \ISA ISA additionally has sixteen 256-bit SIMD registers (\reg{ymm0}--\reg{ymm15}) along with the lower 128-bit sub-register variants (\reg{xmm0}--\reg{xmm15}).
% \cmt{, which are used for floating point and packed operations.}

The \reg{rflags} register stores the current state of the processor.
Specifically, for example, the status flags such as the carry flag (\s{cf}), the parity flag (\s{pf}), the adjust flag (\s{af}), the zero flag (\s{zf}), and the sign flag (\s{sf}) are stored in \reg{rflags}.
These status flags are set according to the result of arithmetic and logical instructions.
% the status flags used mostly in user-level \ISA programs, and updated by arithmetic-logical instructions according to the result of the operation.
Some of control transfer instructions are performed based on the values of these flags.

% Addressing modes: 
\ISA supports the addressing modes, expressions that calculate a memory address to be read or written to. The addressing modes are used as the source or the destination of instructions that access the memory.
The addressing mode expressions can be generalized as:
$\s{base} + \s{index} \times \s{stride} + \s{offset}$.
In the assembly code, for example, \instr{-4(\reg{rax}, \reg{rbx}, 8)} denotes the address mode ``$\reg{rax} + \reg{rbx} \times 8 - 4$''.

% \cmt{      
% \begin{lstlisting}[style=SIMPRULES]
%     D (%RB, %RI, S)
%         RB is register for base
%         RI is register for index (0 if empty)
%         D is displacement (0 if empty)
%         S is scale 1, 2, 4 or 8 (1 if empty)
%     Effective Address: Mem[%RB + D + S*%RI]
% \end{lstlisting}
% }   

% Instruction variants:
% If an instruction access memory, we call it memory instruction. Else, if the instruction takes constant, then it is called immediate instruction. Otherwise, it is a register instructions.
\ISA has three types of instructions depending on the types of their operands: register instructions (with only register operands), memory instructions (with address mode operands), and immediate instructions (with constant operands).
The same mnemonic can be used for the different types of instructions.
For example, the mnemonic \instr{add} can be used for the register instructions (e.g., \instr{addq \reg{rax}, \reg{rbx}}\footnote{Throughout the presentation we will be using the {AT\&T} assembly syntax~\cite{Syntax} where the destination operand comes after source operands.}), the memory instructions (e.g., \instr{addq -4(\reg{rax}), \reg{rbx}}), and the immediate instructions (e.g., \instr{addq \$1, \reg{rbx}}).
% Throughout this paper, we will count each type of instructions as a different instruction.
% In the presentation, we will count them as different insructions ( and call them  instructions variants). Also different opcodes in the Intel manual are counted separately towards total instruction count. 
% \cmt{Whereas, register instructions differing on register names, immediate instructions differing on  constant values and memory instructions   differing on  memory addresses will be counted once towards total instruction count. }




\subsection{K Framework}\label{sec:KF}

\K~\cite{k-primer-2013-v32} \cmt{\url{http://kframework.org}} is a framework for
defining formal language semantics. Given a syntax and a semantics of a language, \K
generates a parser, an interpreter, as well as formal analysis tools such as
model checkers and deductive program verifiers, at no additional cost. Using
the interpreter, one can test their semantics immediately, which significantly
increases the efficiency of semantics developments. Furthermore, the formal
analysis tools facilitate formal reasoning about the given language semantics.
This helps in terms of both applicability of the semantics and
engineering the semantics itself.

We   refer the reader to~\cite{k-primer-2013-v32, rosu-serbanuta-2010-jlap} for
 details. In a nutshell, in \K, a language syntax is given using conventional
Backus-Naur Form (BNF). A language semantics is given as a parametric
transition system, specifically a set of reduction rules over configurations. A configuration is an algebraic representation of the
program code and state. Intuitively, it is a tuple whose elements
(called cells) are labeled and possibly nested. Each cell represents a
semantic component, such as the memory or the registers. A special cell, named \s{k}, contains a
list of computations to be executed. A computation is essentially
a program fragment, while the original program is flattened into a
sequence of computations. A rule describes a one-step transition
between configurations, giving semantics to language
constructs. Rules are modular; they mention only relevant cells that
are needed in each rule, making many rules far more concise and easy to read
than in some other formalisms.



% One of the most appealing aspects of K is its modularity. It is very rarely the
% case that one needs to touch existing rules in order to add a new feature to
% the language. This is achieved by structuring the configuration as nested cells
% and by requiring the language designer to mention only the cells that are
% needed in each rule, and only the needed portions of those cells. For example,
% the above rule only refers to the \s{k} and \s{regstate} cells, while
% the entire configuration contains many more cells (Figure
% ~\ref{fig:config}). This modularity makes for compact and human
% readable semantics, and also helps with the overall effectiveness of the
% semantics development. For example, even if new cells are later added to
% configuration, to support new features, the above rule does not change.

% \begin{figure}[t]
  \centering
  %
  \renewcommand{\dotCt}[1]{\scriptstyle\s{#1}}
  \newcommand{\regstate}{\scriptstyle\s{ID}_\s{regname} \;\mapsto\; \s{Value}}
  \newcommand{\argc}{\scriptstyle\s{Int}}
  \newcommand{\argv}{\scriptstyle\s{Address}} 
  \newcommand{\entrypoint}{\scriptstyle\s{Address}}
%  \newcommand{\labeladdress}{\scriptstyle\textit{ID}_\textit{Label Name} \;\mapsto\; \textit{Address}}
  \newcommand{\codemem}{\scriptstyle\textit{Address} \;\mapsto\; \textit{Instruction}}
  \newcommand{\datamem}{\scriptstyle\s{Address} \;\mapsto\; \s{Value}}
  \newcommand{\bssmem}{\scriptstyle\textit{Address} \;\mapsto\; \textit{Value}}
  \newcommand{\stackmem}{\scriptstyle\textit{Address} \;\mapsto\; \textit{Value}}
  \newcommand{\heapmem}{\scriptstyle\textit{Address} \;\mapsto\; \textit{Value}}
  %
$
\kall{T}{
  \begin{array}{@{}c@{}}
  \kall{k}{\dotCt{K}} \mathrel{}
  \mathrel{}
  \kall{regstate}{\regstate} \mathrel{}
  \mathrel{}
  \kall{memstate} {\datamem} \mathrel{}
  \mathrel{}
  \cmt{ 
  \kall{executionEnv}{ 
  \kall{commandineArgs}{
    \kall{argc}{\argc} \mathrel{}    
    \kall{argv}{\argv} \mathrel{}
  } \mathrel{}
  \kall{entryPoint}{\entrypoint} \mathrel{}
}}
%  \kall{labelAddress}{\labeladdress} \mathrel{}
  \end{array}
}
$
  %
  \caption{Program Configuration}
  \label{fig:config}
\end{figure}


\subsection{Strata Project}\label{sec:prelimstrata}
%% Support Count 
\Strata~\cite{Heule2016a} automatically synthesized formal semantics of \strataWithDupIS{} instruction variants (representing \strataIntel{} unique mnemonics) of the \ISA Haswell ISA. 
%% Set of test input 
The algorithm to learn the formal semantics of an instruction, say \s{IS}, starts with a small set of instructions, called base set \s{B}, whose semantics are known and trusted; a set of test inputs \s{T}, and the output behavior of \s{IS} obtained by executing \s{IS} on \s{T}. Then \Stoke~\cite{Stoke2013} is used  to synthesize  instruction sequences which contain instructions from \s{B} and match the behavior of \s{IS} for all test cases in \s{T}. Given two such generated instruction sequences \s{IS} and \s{IS}$^\prime$, their equivalence is decided  using an SMT solver and the trusted and known  semantics from the base set. If the two sequences are
semantically distinct, then the model produced by the SMT solver is used  to obtain
an input \s{t} that distinguishes \s{IS} and \s{IS}$^\prime$, and \s{t} is added to \s{T}. This process of synthesizing instruction sequence candidates and accepting or rejecting them based on equivalence checking with previous candidates, is repeated until a threshold is reached, which in their implementation is based on the number of accepted instruction sequences. 

%% Generalization 
\cmt{
Using the above technique, they first came up with the semantics of $692$ register
and $120$ immediate instructions. Then they used ``generalization'' of the register
instructions to get a total support count of \strataWithDupIS. Generalization is based on their hypothesis that the memory or immediate variants will behave identically with corresponding register variant (other than where the inputs come from) and hence can use the same formula as register variants. They validate this hypothesis using random testing. }

%% Two format of output
For each instruction, \Strata manifested its semantics in terms of two related artifacts.
The first artifact is an instruction sequence and the second is a set of SMT formulas 
in the bit-vector theory, one for each output register. 
The second is obtained by symbolically executing the first.

\cmt{
\Stoke~\cite{Stoke2013} contains manually written specifications  for a subset of the
x86-64 instruction set and there are APIs to generate SMT formulas from those specifications. 
\Strata uses those formulas to compare against the ones learned
by stratification by asking an SMT solver if the formulas are equivalent. For
example, for the instruction \instr{andnq \%rdx, \%rcx, \%rbx} the manually
written specification looks like the following:

\begin{lstlisting}[style=C++, label={lst:CS5}, caption={Manually Writen  specification for \instr{andnq
\%rdx, \%rcx, \%rbx}}]
semantic_function (
  Operand dst, Operand src1, Operand src2, 
  // dst == %rbx, src1 == %rcx, src2 == %rbx
  SymBitVector a, SymBitVector b, SymBitVector c, 
  // a, b, c are symbolic values for dst, src1, 
  // src2 resp. 
            SymState& ss) {
  // The symbolic state to be updated with the 
  // behaviour of the instruction     
  auto tmp = (!b) & c;
  
  // Setting destinaton
  ss.set(dst, tmp); 
  // Setting of, cf to false 
  ss.set(eflags_of, SymBool::_false()); 
  ss.set(eflags_cf, SymBool::_false());
  // Updating sf, zf based on result
  ss.set(eflags_sf, tmp[dst.size()-1]);
  ss.set(eflags_zf, tmp == SymBitVector::constant(dst.size(), 0));
  // Setting af, pf to undefined values
  ss.set(eflags_af, SymBool::tmp_var());
  ss.set(eflags_pf, SymBool::tmp_var());
}
\end{lstlisting}
}



\section{Formalization of \ISA Semantics}
\label{sec:harvestsema}
This section presents how we get the complete semantics of all the
user-level instructions. Section~\ref{sec:IC} details the scope of our work. Section~\ref{sec:Approach} mentions how we leverage the information available in \Strata, our baseline semantics. Section~\ref{sec:x86sema} explains how we formalize our model in \K.

\subsection{Scope of the Work}\label{sec:IC}
We support all but a few non-deprecated user-level instructions. The support includes  \currentIS{} total variants of the Haswell \ISA ISA (representing \currentIntel{} out of \totalIntel{} 
unique mnemonics). The entire implementation took 8 man-months (not including extensive time spent on projects related to binary decompilation that gave the lead author relevant experience and strong familiarity with the \ISA architecture and documentation).  Below is a summary of the instruction categories that we support:
\begin{itemize}
    \item \textbf{General-Purpose Instructions:} These implement data-movement, arithmetic, logic, control-flow, string operations (including repeated- and fast- string operations).
    
    \item \textbf{Streaming SIMD Extensions (SSE) Instructions \&   subsequent extensions (SSE-2, SSE-3, SSE-4.1, SSE-4.2):} Instructions in this category operate on integer, string or floating-point values stored in $128$-bit xmm registers. Among other things, the category features instructions related to conversions between integer and floating-point values with selectable rounding mode, and string processing.
    
    \item \textbf{Advanced Vector Extensions (or AVX) \& subsequent extensions (Fused-Multiply-Add (or FMA) \& AVX2):} These instructions operate on integer or floating-point values stored in $256$-bit ymm registers; a majority of which are promoted from SSE instruction sets. Additionally, the category features enhanced functionalities specific to AVX \& AVX2, like  broadcast/permute, vector shift, and non-contiguous data fetch operations on data elements. 
    \item \textbf{$\textbf{16}$-bit Floating Point Conversion (or F16C):} These instructions implement conversions between single-precision ($32$-bit) and half-precision ($16$-bit) floating-point values. 
\end{itemize}

Instructions which are \emph{not included} in the current scope of work are: 
(1) System-level instructions, which are related to the operating system, 
protection levels, I/O, cache lines, and other supervisor instructions; 
(2) x87 \& MMX instructions, consisting of legacy floating-point and vector operations, respectively, which are now superseded by SSE; 
(3) Concurrency-related operations, including atomic operations and fences; and 
(4) Cryptography instructions, which support cryptographic processing specified by Advanced Encryption Standard (AES). Note that, there is no inherent limitation in supporting system-level and cryptography instructions with our approach. However, the system-level instructions require to formulate an abstraction of different architectures and operating systems, which is an orthogonal, significant effort that we think should be deserved as an independent work, and we wanted to separate it from the presented effort of formulating the user-level instructions. Nevertheless, K framework makes it easy to add additional state components without modifying rules for operations that do not require those components. We expect our approach to work equally well compared with existing approaches~\cite{Goel:ProCoS17} which implements a subset of system-level instructions. On the other hand, the cryptography instructions are omitted mainly because they are not given a high priority.

\subsection{Overview of the Approach}
\label{sec:Approach:Overview}

Briefly, our approach is as follows.
%
We first defined the machine configuration and underlying infrastructure in the \K framework, in order to define, execute and test the \ISA semantics.
%
To leverage previous work as much as possible, we took all the semantic rules, which amounts to about 60\% of the instructions in scope, from the formal semantics in Strata, in the form of SMT formulas.
%
We corrected, improved or simplified many of the baseline rules.
%
We then translated these SMT formulas from Strata into \K rules using a script, and tested the resulting rules by comparing with the Strata rules using Z3.
%
These steps give us a validated initial set of semantic rules in \K for about 60\% of the target instructions (our ``baseline'' set).

We attempted to extend the stratification approach in Strata to define additional rules automatically, in two ways: (i) augmenting their base set \s{B}, and (ii) constraining the search space manually using knowledge of instruction behaviors.  Both these attempts failed; they worked only for a few instructions, but in general, we found them to be impractical. Specifically, we added $58$ base instructions to the base set, and learned the semantics of $70$ new instructions, which are variants of the added  instructions, in $20$ minutes, but no more even after we kept running for two days. Also, we tried constraining the search space by manually populating it with relevant instructions. The lesson we learned from these experiments is, getting the right set of base instructions or a constrained search space for a complex instruction need an insight about the semantics of that instruction itself. We found that the effort to extract such information from the manual is about the same  as manually defining that instruction.
 

%\Comment{SANDEEP: Please check this last sentence and fix / improve.}

We then manually added \K rules for the remaining 40\% of the target instructions by systematically translating their description of the Intel manual into \K rules, in some cases cross-referencing against semantics available in Stoke.
%
The outcome was a complete formal specification of user-level \ISA in \K.

We validated this semantics in three ways, as described in Section~\ref{sec:Eval}.
%
First, we use the \K interpreter to execute the semantics of \emph{each} instruction for 7,000+ test inputs (each input is a processor state configuration) and compared the output directly with the hardware behavior for the same instruction.
%
Second, we repeated this experiment using the applicable programs in the GCC torture tests.  % (\TortureInclude{} out of \TortureTotal{} tests).
%
Third, we compared against the semantics defined in the Stoke project for about 330 instructions that were omitted in Strata (thus not included in our baseline), using an SMT solver.

These validation experiments uncovered bugs in the Intel manual, in Strata's simplification rules, and in the Stoke semantics.  All these bugs were reported to the authors, and most have been acknowledged and some have been fixed.  The details are in Section~\ref{sec:Eval}.



% \section{Defining Semantics of \ISA ISA in K}\label{sec:x86sema}

% This section is about defining the semantics of individual instructions in \K along with  execution environment, which is used to execute instructions using \K framework. \cmt{We describe the different components of our definition and give a number of example of the semantics.}

% \cmt{ \subsection{Syntax}
% Our model parsers the syntax of \ISA assembly language program which is generated either from the source code using GNU assembler~\cite{GNUAS} or
% directly from the binary using disassemblers like Objdump~\cite{Objdump}.  
% }

\subsection{Program Configuration}\label{sec:x86sema}

\begin{figure}[t]
  \centering
  %
  \renewcommand{\dotCt}[1]{\scriptstyle\s{#1}}
  \newcommand{\regstate}{\scriptstyle\s{ID}_\s{regname} \;\mapsto\; \s{Value}}
  \newcommand{\argc}{\scriptstyle\s{Int}}
  \newcommand{\argv}{\scriptstyle\s{Address}} 
  \newcommand{\entrypoint}{\scriptstyle\s{Address}}
%  \newcommand{\labeladdress}{\scriptstyle\textit{ID}_\textit{Label Name} \;\mapsto\; \textit{Address}}
  \newcommand{\codemem}{\scriptstyle\textit{Address} \;\mapsto\; \textit{Instruction}}
  \newcommand{\datamem}{\scriptstyle\s{Address} \;\mapsto\; \s{Value}}
  \newcommand{\bssmem}{\scriptstyle\textit{Address} \;\mapsto\; \textit{Value}}
  \newcommand{\stackmem}{\scriptstyle\textit{Address} \;\mapsto\; \textit{Value}}
  \newcommand{\heapmem}{\scriptstyle\textit{Address} \;\mapsto\; \textit{Value}}
  %
$
\kall{T}{
  \begin{array}{@{}c@{}}
  \kall{k}{\dotCt{K}} \mathrel{}
  \mathrel{}
  \kall{regstate}{\regstate} \mathrel{}
  \mathrel{}
  \kall{memstate} {\datamem} \mathrel{}
  \mathrel{}
  \cmt{ 
  \kall{executionEnv}{ 
  \kall{commandineArgs}{
    \kall{argc}{\argc} \mathrel{}    
    \kall{argv}{\argv} \mathrel{}
  } \mathrel{}
  \kall{entryPoint}{\entrypoint} \mathrel{}
}}
%  \kall{labelAddress}{\labeladdress} \mathrel{}
  \end{array}
}
$
  %
  \caption{Program Configuration}
  \label{fig:config}
\end{figure}


Defining a language semantics in \K requires defining the program configuration, the semantics of how programs are evaluated (i.e., the execution environment), and the semantics of the statements or instructions.  We begin with the configuration.

The \K configuration of a running \ISA program is shown in Figure~\ref{fig:config}. The cells are represented using angle brackets. The outer $\top$ cell contains the cells used during program evaluation. The inner \s{k} cell contains a list of computations to be executed. Below we describe the two other inner cells.

\paragraph{Register State}

The \s{regstate} cell contains a  map from registers or flag names to values. The register names include the sixteen general purpose registers, \reg{rip}, and the sixteen SIMD registers.  The value mapped to a register name is a $64$-width bit-vector (or a 256-width one for the SIMD registers). Values for sub-register variants are derived from the register values by extracting the relevant bits. \cmt{In the \s{regstate} map, }We store individual flag names (mapped to a bit-vector value of width $1$) as opposed to a $64$-bit \s{rflags} register. Every access (read/write) of \reg{rflags} retrieves the entries in the \s{regstate} map for the individual flags.

\paragraph{Memory State}

The \s{memstate} cell is a map from $64$-bit addresses to bytes, which specifies the byte-addressable memory, but our implementation is flexible enough to use alternative memory  representations with addressing of $2$-byte or $4$-byte quantities.  Our memory layout is ``flat'', in which all available memory locations can be addressed, \cmt{without going through additional levels of addressing, such as memory segments.} but we do have logical partitions\footnote{These abstractions are useful for executing \ISA programs.} of the memory into sections like code, data and stack.  The following is an example snapshot of a memory state,  holding a $4$-bytes integer value $65535$:
\begin{center}
    \newcommand{\bytecell}[2]{\s{byte}({#1},\;\;{#2})}
    $
    \kall{memstate} {
        %\begin{array}{@{}c@{}}
        \begin{array}{ll}
        4 \;\mapsto\; & \bytecell{0}{32'65535}\\
        5 \;\mapsto\; & \bytecell{1}{32'65535}\\
        6 \;\mapsto\; & \bytecell{2}{32'65535}\\
        7 \;\mapsto\; & \bytecell{3}{32'65535}\\
        \end{array}
    }
    $
\end{center}
This says that the memory address $4$ stores the $0^\text{th}$ byte of the bit-vector \bv{32}{65535}\footnote{All the values and address are represented as bit-vectors which are depicted in this paper  as \bv{W}{V}, and interpreted as a bit-vector of size $W$ and value $V$.}, the address 5 stores the $1^\text{st}$ byte, and so on. While memory read, requested bytes are aggregated according to the size of the memory access.



\subsection{Semantics of Execution Environment}
%\begin{figure*}[t]
\begin{tabular}{l}      
\begin{lstlisting}[basicstyle=\scriptsize,morekeywords={andBool, requires, ensures, and, rule},keywordstyle=\color{blue},escapeinside={(*}{*)},]
rule <k>  (OpC:Opcode OpR:Operands):Instruction) Rest:Instructions ~> fetch </k>
\end{lstlisting} \\

\begin{lstlisting}[basicstyle=\scriptsize,morekeywords={andBool, requires, ensures, and, rule},keywordstyle=\color{blue},escapeinside={(*}{*)},]
rule  <k> OpC:Opcode OpR:Operands => (*\textbf{.}*) ...</k>
   <memstate>... L <- storedInstr(OpC OpR) ...</memstate>
   <nextLocPc> L:Address => L + sizeofInstr(OpC OpR) </nextLocPc>
\end{lstlisting} \\

\begin{lstlisting}[basicstyle=\scriptsize,morekeywords={andBool, requires, ensures, and, rule},keywordstyle=\color{blue},escapeinside={(*}{*)},]
rule <k> fetch => execinstr(OpC OpR) ~> fetch ... </k>
  <memstate>... L |-> storedInstr(OpC OpR) ...</memstate>
  <regstate>... "RIP" |-> ( L => sizeofInstr(OpC OpR))  ...</regstate>
\end{lstlisting} \\

\begin{lstlisting}[basicstyle=\scriptsize,morekeywords={andBool, requires, ensures, and, rule},keywordstyle=\color{blue},escapeinside={(*}{*)}]
rule <k> fetch => (*\textbf{.}*)  ... </k>
  <memstate> codeMap:Map  </memstate>
  <regstate> "RIP" |-> BV(*$_\s{rip}$*) </regstate>
    requires not iloc( {RSMap["RIP"]}:>MInt ) in_keys ( CMap )
\end{lstlisting} \\


\end{tabular}
        \caption{Semantics of Execution Environment}
        \label{fig:compSema}
\end{figure*}


\cmt{\begin{figure*}[t]

$
\begin{array}{@{}l@{}}

\mbox{\scshape {\scriptsize rule}: \textbf{\scriptsize inital value of k cell}} \\
\kall{k}{\scriptsize {\tt initEnv}(\$ARGV)\kra ((OpC{\tt:Opcode}\;\; OpR{\tt:Operands}){\tt:Instruction} \;\;Rest{\tt:Instructions})\kra{\tt fetch}} \\
\multicolumn{1}{c}{(a)} 
\\[0.3cm]

\mbox{\scshape {\scriptsize rule}: \textbf{\scriptsize load instruction}} \\
\kall{k}{\scriptsize \reduce{OpC{\tt:Opcode}\;\; OpR{\tt:Operands}}{\cdot} \ellipses}
\mathrel{}
\kall{memstate}{\scriptsize \ellipses \reduce{\cdot}{BV_{L} \;\mapsto\; (OpC\;\;OpR)} \ellipses}
\mathrel{}
\kall{nextloc}{\scriptsize  \reduce{BV_{L}}{BV_{L} + {\tt sizeof}(OpC\;\;OpR)}}
\\
\multicolumn{1}{c}{(b)} 
\\[0.3cm]

\mbox{\scshape {\scriptsize rule}: \textbf{\scriptsize fetch instruction}} \\
\kall{k}{\scriptsize \reduce{{\tt fetch}}{{\tt exec}(OpC\;\;OpR) \kra fetch} \ellipses}
\mathrel{}
\kall{memtate}{\scriptsize \ellipses BV_{L} \;\mapsto\; (OpC\;\;OpR) \ellipses}
\mathrel{}
\kall{regstate}{\scriptsize \ellipses RIP \;\mapsto\; \reduce{BV_{L}}{BV_{L} + {\tt sizeof}(OpC\;\;OpR)} \ellipses}
\\
\multicolumn{1}{c}{(c)} 
\\[0.3cm]

\mbox{\scshape {\scriptsize rule}: \textbf{\scriptsize program termination}} \\
\kall{k}{\scriptsize \reduce{{\tt fetch}}{\cdot} \ellipses}
\mathrel{}
\kall{memstate}{\scriptsize codeMap{\tt :Map}}
\mathrel{}
\kall{regstate}{\scriptsize \ellipses RIP \;\mapsto\; BV_{rip}\ellipses}
\\
  \qquad \texttt{when notBool}\;\;BV_{rip}\;\; \texttt{in\_keys}\;\;codeMap 
  \mathrel{}
\\
\multicolumn{1}{c}{(d)} 
\\
\hline
\end{array}
$

%
\caption{Semantics of Execution Environment}
\label{fig:compSema}
\end{figure*}
}


We now give the reader a flavor of our semantics, by discussing a few of the roughly $5,200$ rules that we defined to model the entire semantics.
%\Comment{Can you list how many of these rules came from Strata unchanged, 
%how many had to be "fixed" in various ways, how many you wrote yourself?}
We first explain the semantics of the execution environment, which involves all the machinery used for executing \ISA programs.
We will explain the semantics of individual instructions in the next section.

The semantics of execution of an \ISA program involves initializing the memory with program instructions and then  fetching the instructions from the memory one at a time and executing it. The instruction to be executed next is pointed to by the instruction pointer register \reg{rip}.  

The following rule is applied to initialize memory with instructions one at a time: 
\begin{lstlisting}[style=KRULE]
rule  <k> OpC:Mnemonic OpR:Operands (*$\Rightarrow$*) (*\textbf{.}*) ...</k>
  <memstate> M:Map (*$\Rightarrow$*) M[L (*$\leftarrow$*) (OpC OpR)] </memstate>
  <nextloc> L:Address (*$\Rightarrow$*) L + instrSize(OpC OpR) </nextloc>
\end{lstlisting}
The \s{k} cell contains the instruction to be processed next. We use ``\s{:T}'' to represent the type of a term used in a rule. For example, \s{Mnemonic} and \s{Operands} denote the types of the terms used to represent an instruction.  The `$\Rightarrow$' symbol
represents a reduction (i.e., a transition relation). A cell without the `$\Rightarrow$' symbol means that it is read but not changed by the rule. \K allows us to use ``.'' to represent an empty computation and ``...'' to represent a structural frame, that is, it matches the portions of a cell that are neither read nor written by the rule. The above rule essentially  stores each instruction in memory, which is modeled as a map, at an address \s{L} given by the \s{nextloc} cell. Subsequently, the \s{nextloc} cell gets updated to an appropriate address  used for storing the next instruction.  Once the entire program is loaded, the fetch-and-execute cycle starts, which is realized by the following rule:
\begin{lstlisting}[style=KRULE]
rule <k> fetch (*$\Rightarrow$*) exec(OpC OpR) (*$\kra$*) fetch ... </k>
  <memstate>... L (*$\rightarrow$*) (OpC OpR) ...</memstate>
  <regstate>... 
    "RIP" (*$\rightarrow$*) ( L (*$\Rightarrow$*) L + instrSize(OpC OpR))  
  ...</regstate>
\end{lstlisting}
%\Comment{It is not clear what kind of term \Code{exec(OpC OpR)} is: is it a special function defined somewhere, or a separate rule? I'm not familiar enough with K, but remember that reviewers won't be familiar either!}
%
 The symbol $\kra$ is used to separate the computations in the \s{k} cell. The rule reads the instruction, pointed to by the instruction pointer register \reg{rip}, from memory and puts it at the head of the \s{k} cell along with the ``fetch'' computation to be executed in order and updates the value of \reg{rip}. The execution will be terminated when there is no instruction stored in the memory at the address pointed to by \reg{rip}.\footnote{While initializing the stack section of memory, we store an invalid address just before the entry-point function as return address. When the entry point function returned, the invalid return address is popped out of the stack and stored in \reg{rip} leading to program termination.} 

\subsection{Semantics of Individual Instructions}
Here we explain how we define the semantics of an instruction in \K using a running example of logical-and-not \instr{andnq -4(\%rsp), \%rbx, \%rax}, which performs a bitwise logical AND of inverted source register operand (\reg{rbx}) with the source memory operand (\mem{-4}{rsp}) and writes the result to destination register \reg{rax}. Additionally the instruction affects all the $6$ status flags.


%\begin{figure*}[]
    \begin{tabular}{ll} 
        $
       \begin{array}{@{}l@{}}
          \kprefix{k}{\reduce{{\tt exec(pushw\;\;4(\%rsp)\;\;.List)}}
              {{\tt 4(\%rsp)} \kra {\tt exec(pushw\;\;\Box\;\;.List)}}} \mathrel{}
          \mathrel{}  
          \\
           \multicolumn{1}{c}{(a)} 
       \end{array}   
       $               
        & 
       $
        \begin{array}{@{}l@{}}
           \kprefix{k}{\reduce{{\tt 4(\%rsp)}}{{\tt memOffset( BV_{rsp} + 64'4)}}} \mathrel{}
        \kall{regstate}{\ellipses{\tt RSP}\mapsto {\tt BV_{rsp}}\ellipses} \mathrel{} \mathrel{}
       \\
       \multicolumn{1}{c}{(b)} 
       \end{array}  
       $ \\   \\ [1pt]
 
 $
 \begin{array}{@{}l@{}}
  \kprefix{k}{\reduce{
          {\tt memOffset(O)} \kra {\tt exec(pushw\;\;\Box\;\;.List)}}
      { {\tt exec(pushw\;\;memOffset(O)\;\;.List)}}} \mathrel{} 
 \\
 \multicolumn{1}{c}{(c)} 
 \end{array}   
 $               
 & 
 $
 \begin{array}{@{}l@{}}
 \kprefix{k}{\reduce{
         {\tt exec(pushw\;\;memOffset(O)\;\;.List)}}
     { {\tt readFromMem(O, 16) \kra exec(pushw\;\;memOffset(O)\;\;.List)}}} \mathrel{} 
 \\
 \multicolumn{1}{c}{(d)} 
 \end{array}  
 $ \\ \\ 

     
  \multicolumn{2}{c}{
  $
  \begin{array}{@{}l@{}}
  \kprefix{k}{\reduce{
          {{\tt {memValue(V) \kra exec(pushw\;\;memOffset(O)\;\;.List)}}}
      }
      {{\tt storeToMem(V,\;\;getRegisterVal(\%rsp) - 64'2,\;\;16)}}} \mathrel{} 
  \kall{regstate}{\ellipses{\tt RSP}\mapsto \reduce{{\tt BV_{rsp}}}{{\tt BV_{rsp}} - 64'2}\ellipses} \mathrel{} \mathrel{}
  \\
  \multicolumn{1}{c}{(e)} 
  \end{array}
  $      
    }\\ \hline
    \end{tabular}
    \caption{Semantics of \s{pushw (\%rsp)}}
    \label{fig:pushw}
\end{figure*}
















The semantics of most of the instructions can be modeled broadly in $3$ phases: (1) read the data from source operand(s), which could be a register, memory or constant value; (2) operate on the data based on the mnemonic; and (3) write the result(s) to destination operand(s), which could be a register or memory. An instruction may exercise some or all of the above phases. 
   
\paragraph{Read from Source Operand(s)}
Instruction in the running example reads from  register (\reg{rbx}) and memory (\mem{-4}{rsp}) operands. A read from register is modeled as a lookup with register name in the \s{regstate} map and subsequent read of the mapped value or,
for a sub-register, a portion of it. The semantics of register read can be defined as:  
\begin{lstlisting}[style=KRULE]
rule <k> getRegisterVal(R:R64) (*$\Rightarrow$*) BV(*$_\s{r}$*) ...</k>
     <regstate>... "R" (*$\mapsto$*) BV(*$_\s{r}$*) ...</regstate>
\end{lstlisting}
 In the context of the running example, this rule
is applied when the current computation (at top of the \s{k} cell) is a $64$-bit register
lookup, depicted as  \s{getRegisterVal(\reg{rbx})}, and \s{regstate} contains a register
with name ``RBX''. This rule resolves the register lookup  to the mapped bit-vector value \s{BV$_\s{r}$}. 

A read from memory involves computing the effective address in the memory, looking-up that address in memory, and reading requested bytes from memory if the memory access is within allowed range.
The following rule is applied to compute the effective address:
\begin{lstlisting}[style=KRULE]
rule <k> 
 Offset:Int (R:R64):Mem (*$\Rightarrow$*) ( 64'Offset + BV(*$_\s{r}$*)):Address
...</k>
  <regstate> ... "R" (*$\mapsto$*) BV(*$_\s{r}$*) ... </regstate>
\end{lstlisting}
The term to the left of $\Rightarrow$ shows the memory addressing expression at the 
top of \s{k} cell, which gets reduced to a memory address. The effective memory address for the memory operand used in the running example is (64'-4 + \s{BV$_\s{rsp}$}) which is then used to do memory read access. 
The rules associated with memory read is responsible to read a memory value of requested number of bits ($64$-bits for the current example) starting from the effective memory address. 
 
\paragraph{Operate on Data}
The rules for operating on operands will be different for each instruction based on the mnemonic. For example, the mnemonic \instr{andnq} requires logical-and-not operation to be computed on the operands.

\paragraph{Write to Destination Operand(s)}
The example instruction writes the result to a destination register \reg{rax}. Also, the flags \s{sf} and \s{zf} are updated based on the result; \s{of} and \s{cf}  are cleared, and \s{af} and \s{pf} are undefined. The following rule realizes the destination write operation, where \s{memVal$_{64}$} and \s{BV$_\s{r}$} represents the $64$-bit data values evaluated using the respective rules for register and memory operands (mentioned above).
%
%\Comment{Should "R" be "Dest" below in the first regstate rule?}
%
\begin{lstlisting}[style=KRULE]
rule <k>
  exec(andnq memVal(*$_{64}$*), BV(*$_\s{r}$*), R:R64) (*$\Rightarrow$*) (*\textbf{.}*)
...</k>
  <regstate>
    "R"  (*$\rightarrow$*) _ (*$\Rightarrow$*) ((*$\sim$*)BV(*$_\s{r}$*) & MemVal(*$_{64}$*))
    // sf and zf are updated based on the result. 
    "SF" (*$\rightarrow$*) ((*$\sim$*)BV(*$_\s{r}$*) & MemVal(*$_{64}$*))[63:63]
    "ZF" (*$\rightarrow$*) ((*$\sim$*)BV(*$_\s{r}$*) & MemVal(*$_{64}$*)) (*$=$*) 64'0 ? 1'1:1'0
    // of and cf are cleared. 
    "OF" (*$\rightarrow$*) 1'0
    "CF" (*$\rightarrow$*) 1'0
    // sf and zf are undefined.
    "AF" (*$\rightarrow$*) undef
    "PF" (*$\rightarrow$*) undef
  ...</regstate>  
\end{lstlisting}
The operator ``[i:j]'' extracts bits i down to j from a bit-vector of size n, yielding another  bit-vector of size i - j + 1, assuming that n $>$ i $\ge$ j $\ge$ 0. The operator ``$\&$'' implements bit-wise and operation. The rule associated with memory write is similar to that for memory read and is skipped here. 

A \ISA program is modeled as a list of instructions and its semantics is  given by composing the semantics of its constituents. The reader is encouraged to take a look at the instruction semantics (made available as supplemental material~\cite{Suppl}).

\subsection{Constructing the \ISA Semantics}
\label{sec:Approach}

% we built our complete semantics on top of the \Strata semantics that we took as our baseline.
%
% As mentioned in section~\ref{sec:prelimstrata}, 

\paragraph{Systematic Translation of \Strata Rules to \K}
%
As mentioned in the introduction, we leverage the \Strata~\cite{Heule2016a} semantics to develop our complete semantics, to minimize the overall effort.
We systematically translated their semantics into \K.
%
Specifically, \Strata offers the semantics of \strataWithDupIS{} instruction variants as SMT formulas specifying the behavior of output registers.
For each instruction, we converted the SMT formulas that \Strata provides to a \K specification using a simple script ($\sim$500 LOC).
% The ease of this porting is facilitated by the fact that SMT-LIB expressions use s-expression (or "symbolic-expression") notation ~\cite{Steele} which eases the parsing effort significantly.

To validate the translation, we generated SMT formulas from the translated \K specifications (using APIs provided by the \K framework), and use the Z3 SMT solver to check their equivalence to the corresponding formulas provided by \Strata.
% In the course of the validation, we found bugs in \Strata's SMT formula extraction module.
%
While translating and validating their semantics, we found various issues that we had to fix to establish our baseline semantics.
%
Below we describe the issues we found in \Strata.%, and how we fixed them  as follows.

% as we explain our general methodology of the semantics formalization.
% Our approach is to borrow this available information and see if we can use it towards faithful modeling of instruction semantics.
% Modeling the semantics of remaining \currentManual{} (\currentIS{} - \strataWithDupIS{}) instructions are discussed in section~\ref{sec:US}. 




% \paragraph{Challenges in Modeling Semantics of \ISA ISA} 

% Fully modeling the semantics of \ISA ISA is more challenging than it would
% seem. In order to appreciate the difficulty, we will walk through few examples.

% \paragraph{Status Flags}

% Consider the following \ISA code snippet that features
% \emph{Logical And Not} instruction \instr{andnq}, the semantic of which is
% mentioned as a side comment. The following instruction \instr{jz target}, makes a conditional jump to \TS{target} if the result of the previous operation is $0$ (which is manifested by a $0$ value in the \reg{zf} flag), otherwise the
% control-flow  will reach the following instruction. 
% \begin{lstlisting}[style=Bash, caption={A code snippet in \ISA Assembly Language },label={lst:CS1}]
% andnq %rax, %rbx, %rcx  
%         # %rcx = (~(%rbx) bitwiseAnd %rax)
%         # %sf, %zf: Updated based 
%         #           on result %rcx
%         # %of, %cf: Set to 0
%         # %af, %pf: Undefined

% jz target
%         # jump to target 
%         # if zf == 1
% ...     # following instruction
% \end{lstlisting}

% Suppose we want to find inputs value to \reg{rax} and \reg{rbx} which will
% trigger the control-flow to either directions. We can do a symbolic execution
% of the code snippet using symbolic input value to \reg{rax} and \reg{rbx} and
% later using a theoram prover like \Z, we can solve the path condition created
% at line 8, to get the required concrete inputs.  The operation \emph{Logical
%   And Not}  sets up to $6$ status flags, and control flow  depends upon the
%   values of those flags, not directly on the result of the operation. Hence, a
%   simple model of the first instruction to something like \emph{rcx =
%     $\sim$(rbx) and rax} does not expose those side effects and hence the
%     symbolic execution will fail to reason about the program.  The following
%     shows a correct modeling of the \reg{zf} flag for \instr{andq \%rax, \%rbx,
%       \%rcx} (in  SMT-LIB 2.0\cite{smtlib}).

% \begin{lstlisting}[style=SMTLIB, caption={A code snippet in \ISA Assembly Language },label={lst:CS2}]
% /* 
% ** If the value of ~(%rbx) and %rax is 0, 
% **  return bit 1 or bit 0 
% */
% (ite 
%     (= 
%       (bvand 
%         (bvnot %rbx) 
%         %rax
%       ) 
%       #x0000000000000000
%     ) #b1 #b0)
% \end{lstlisting}

% A symbolic execution using the above model of \reg{zf} will provide us with
% path condition 
% $
% \text{(\textbf{=} (\textbf{bvand} \%rax (\textbf{bvnot} \%rbx)) \#x0000000000000000)}
% $
% \\
%  for which the target path is taken, which can then be solved to get a
%  triggering input value. Similarly solving the negation of the above path
%  condition will provide input to trigger the other path.




% \paragraph{Undefined Behaviors}
% % \paragraph{Undefined, Implementation-Dependent Behaviors}
% 
% According to the \ISA standard, many instructions (\undefIntel{}\footnote{\label{note1}These numbers are obtained by parsing the official manual ``Volume 2: Instruction Set Reference'' and cross checked with projects~\cite{Stoke2013, Felix} investing similar efforts.} out of \totalIntel{})
% have undefined, implementation-dependent behaviors: their output values of the destination register or the \reg{rflags} register are undefined in certain cases.
% In our semantics, we faithfully modeled the undefined value as a unique symbol (say, \s{undef}) whose value is nondeterministically decided within the proper range.
% These nondeterministic values are enough to capture and formally reason about all possible behaviors of the instructions for different processors (and even any future processor that conforms to the standard).
% Regarding the instruction-level testing (Section~\ref{sec:Eval}), we consider the \s{undef} symbol to be matched with any concrete value provided by the hardware, so that we can test the instructions modulo the undefined behaviors.
% 
% We found that other semantics do not faithfully model the undefined behaviors, simply following a specific behavior taken by a processor implementation.
% %
% For example, the parity flag \reg{pf} is undefined in the logical-and-not instruction \instr{andn}, where the processor implementation is allowed to either update the flag value (to 0 or 1), or keep the previous value.
% We found that Strata~\cite{Heule2016a} updates the flag based on the result of the \instr{andn} operation, whereas a binary analysis tool Radare~\cite{Radare2}, for example, keeps it unmodified.



% Vol. 1. APPENDIX A EFLAGS CROSS-REFERENCE
%%636 = 474 + 162; 162 are all included in 474 
% There are \undefTotal{}\footnote{\label{note1}These numbers are obtained by parsing the official manual ``Volume 2: Instruction Set Reference'' and cross checked with projects~\cite{Stoke2013, Felix} investing similar efforts.} instructions that places undefined value to a flag register either conditionally based on some condition of the input state (register or memory) or un-conditionally. We ensured to model all those flags by consulting the Intel Manual.

% For conditional undefined cases, we 
% tested against hardware with
% input values which does not trigger the
% condition for undefined-ness.  For the remaining cases; (1) where the input value triggers the condition for undefined-ness 
% (2) where the instruction places undefined value to output register irrespective of input value, we make sure that the output registers hold \udef token after \K execution.



%
% Consider a slightly different code snippet where the conditional jump is based
% the value of undefined flag\footnote[1]{Even though a compiler does not seem to
% produce such a code snippet, but we can realize a code injection attack using
% the code snippet whose purpose is to jump on  a malicious \emph{target} on
% a specific processor implementation.}, \reg{pf}. 
% \begin{lstlisting}[style=Bash]
% andnq %rax, %rbx, %rcx       
% jp target
%       # jump to target 
%       # if pf == 1
% ...    # following instruction 
% \end{lstlisting}
%
% Some binary analysis tools like Radare~\cite{Radare2} keeps the flag
% unmodified, whereas Strata~\cite{Heule2016a} chose to compute the parity flag
% based on the result of the \instr{andn} operation.
% In effect, these
% approaches restricts the value of the flag to a concrete value and hence a
% semantics model based on these approaches prevent exploration of paths feasible
% in some processor implementation.  Therefore, it is desirable to identify these
% cases where a register or a flag could be undefined and to encode this
% information in the model in such a way so as to assist exploration of all
% resulting execution paths.



\paragraph{Status Flags}

We found that \Strata omitted to specify the \reg{af} flag behaviors, as the flag is not commonly used.
However, we faithfully specified the semantics of all the status flags in the \reg{rflags} register, even if some of them are not commonly used, since they may affect the overall program's behavior in some tricky cases, and we do not want to miss any of such details when formally reasoning about the \ISA programs.

% We had to revise the semantics of $689$\footnoteref{note1} instructions that affect the \reg{af} flag, when we borrow the \Strata semantics as our baseline.




\paragraph{Instruction Variants}

\Strata essentially provides the semantics of the register instructions, assuming that the semantics of the memory and immediate instruction variants can be obtained by generalizing the register instructions.
However, we found that certain memory instructions cannot be inferred by simply generalizing their corresponding register instructions.
%
For example, for \instr{movsd}, one of the 128-bit SSE instructions, its register variant has quite different semantics from the memory variant.
Below are their pseudo-code semantics:
{
\small
\begin{center}
\begin{tabular}[b]{l}
   // Register Variant: movsd \%xmm1  , \%xmm0 \\ 
   {\tt S1.} XMM0[63:0] $\leftarrow$ XMM1[63:0] \\ 
   {\tt S2.} XMM0[127:64] (Unmodified) \\
   \\
   // Memory Variant: movsd  (\%rax) , \%xmm0 \\
   {\tt S1.} XMM0[63:0] $\leftarrow$ MEM\_ADDR[63:0] \\
   {\tt S2.} XMM0[127:64] $\leftarrow$ 0
\end{tabular}
\end{center}
}
\noindent As seen, only the memory instruction clears the higher 64 bits of the destination register, which cannot be inferred from the register instruction behavior that does not touch the higher bits at all.
%
We found that another 128-bit SSE instruction, \instr{movss}, has the same generalization issue.
%
For the other instructions, we obtained the memory and immediate variants by generalizing the register variants, and validated the generalization by co-simulating the inferred semantics against a processor.


\paragraph{Immediate Instruction Variants}

There are 118 immediate instruction variants (over the 8-bit constants) that do not have corresponding register instructions.
For those immediate instructions, Strata provides the instruction semantics for each individual constant, resulting in 30,208 (= $118 \times 256$)\footnote{Indeed, Strata provides explicitly only 19,783 formulae by randomly sampling $\sim$168 constants out of 256, in average, for each immediate instruction, assuming that the remaining 10,425 formulae can be inferred.} formulae for the immediate instructions' semantics.
We generalized the set of formulae for each immediate instruction into a single semantic rule.
We validated our generalization by cross-checking the generalized semantics with the original using the SMT solver.

% Moreover, there are $118$ immediate variants with $8$-bit constant value, which
% do not have a corresponding register variant to generalized from. \Strata tries to learn $256$ separate formulas for every possible value of
% the $8$-bit immediate operand. We would like the instruction semantics to be concise and maintainable.  For each of those 
% instructions, we manually write  its semantic in \K and then use APIs provided by \K framework to get the
% corresponding SMT formula. We then compare each of those separate formulas, for a
% constant say \s{c}, with our SMT formula by replacing its symbolic inputs  with
% the constant value \s{c}. Upon successful equivalence check, for all values of
% \s{c}, we accept the manually written semantics. We managed to get a concise
% semantics for all of the $118$ immediate variants. \cmt{Note that, had we used \Strata's  formula with a $k$-bit immediate, \Strata needs 2$^k$ formulas each of which has to be checked, but we have $1$ formula which has to be checked 2$^k$ times, so testing cost is similar but final result is exponentially more concise.}
% %         works for all values of the immediate operand) for those instructions
% %     because that way is more intuitive and easy to maintain.  




% the the semantics of  memory or immediate instruction variants which are obtained from the register variant using ``generalization'' by co-simulating it  against a physical x86 machine.

% While doing that,  we found instructions like \instr{movsd (\%rax), \%xmm0} where the generalization from the corresponding register variant, \instr{movsd \%xmm1, \%xmm0} is not faithful. Followings are their specification as per the Intel manuals.

% Clearly, the memory variant cannot be obtained using generalization of the  corresponding register instruction. Same is true with instruction \instr{movss (\%rax), \%xmm0}.   


%\footnote{Even though \Strata have simplification rules to prevent formula blowup but we found that they are not adequate for many instructions. Hence we added $13$ additional rules.}
 
\paragraph{Formula Simplification}

Due to the nature of the stratification, \Strata provides complex formulae for certain instructions.
%
We simplified those complex formulae by either applying some simplification rules or manually translating into simpler ones.
Then we validated the simplification by checking the equivalence between them using the SMT solver.
%
For example, the original \Strata-provided formula for \instr{shrxl \%edx, \%ecx, \%ebx} consists of 8971\cmt{2560} terms (including the operator symbols), but we could simplify it to a formula consisting of only 7 terms.\footnote{Refer to our semantics~\cite{Suppl}, specifically \url{instruction-semanatics/shrxl_r32_r32_r32.k}, for details.}




% We wrote several simplification rules to automatically simplify the complex formulae.
% For the complex formulae that are not simplified enough by the simplification rules, we manually translated them into simple, but equivalent formulae, and check the equivalence 


% We added additional $13$ simplification rules in \Strata framework to simplify \Strata formulas even further. This helps in getting rid of many redundant \uif.

% After applying the simplification rules, we checked the \Strata formulas\footnote{Note that we have to check the formulas of only the register variants which are generated by \Strata as the rest are obtained either by``generalization'', in which case, the formulas for memory and immediate variants are guaranteed to be simple if the register variants are simple, or has been already written manually in simplified form.} if they are still un-simplified. In that case, we write the specification of that instruction in \K (similar to Figure~\ref{fig:pushw}) and compare the SMT formula, obtained by using \K APIs, with the \Strata formula for equivalence. .

% Following are some of them:
% \begin{lstlisting}[style=SIMPRULES]
% /* Assume 
%     A, B, C are symbolic bit-vector values of 
%         width W,
%     W'N denotes a constant bit-vector value of 
%         width W and integer value N.    
%     Cond is a symbolic boolean value, and  
%     I, J and K are symbolic integers s.t.  
%         K (*$\geq$*) J (*$\geq$*) I (*$\geq$*) 0.
    
%   A (*$\circ$*) B  denote bit-vector concatenation. 
%   A[J:I] denotes bitvector-extract operation. 
%   '+'    denote addition over bit-vectors
%   '(*$\otimes$*)' denote  any of xor, or, add operators.
% */
% // Emilimitae redundant uninterpreted functions.
%   (*$\bullet$*) (*$\s{add\_double(W'0, A)}\;\;\equiv\;\;\s{A}\;\;\text{{\color{blue} if}}\;\; \s{A}\equiv$\;\;\text{\s{concat(W1'0, W2'N) s.t. W1 + W2 = W}}*) 
% // Distribute over if-then-else
%   (*$\bullet$*) (*\s{(Cond ? A : B)[J:I]} $\equiv$ \s{(Cond ? A[J:I] : B[J:I])}*)
%   (*$\bullet$*) (*\s{X} $\circ$ \s{(Cond ? A : B)} $\equiv$ \s{(Cond ? X$\circ$A : X$\circ$B)}*)
%   (*$\bullet$*) (*\s{(Cond ? A : B)} $\otimes$ \s{(Cond ? C : D)} $\equiv$ \s{(Cond ? A $\otimes$ B : C $\otimes$ D)}*)
% // Distribute extract over addition.
%   (*$\bullet$*) (*\s{(A + B)[J:0]} $\equiv$ \s{A[J:0] + B[J:0]}*)
% // Merge consecutive  concatenations.  
%   (*$\bullet$*) (*\s{(A[K][J] $\circ$ (B[J][I] $\circ$ C))} $\equiv$ \s{A[K:I] $\circ$ C}*)  
  
% \end{lstlisting}






% \paragraph{Challenges Borrowing Semantics from Strata}
% At first glance it might seem that the \cmt{straightforward to use} \Strata generated artifacts as useful and  faithful to be used directly as the formal semantics, but our experience tells a
% different story and  following are the reasons:

% \begin{enumerate} 
%     \item \textbf{Non-Intuitive Formula for Stratified Immediate Instruction }
%     \Strata learns the semantics of $120$ immediate variants whose immediate operand is of size $8$ bits. The reason that they
%     do not use ``generalization'' (as mentioned in section~\ref{sec:prelimstrata}) is that these instructions do not have a
%     corresponding register-only variant to generalized from.  In all fairness,
%     \Strata tries to learn a separate formula for every possible value of the
%     $8$-bit immediate operand.  We intend to have a ``concise'' semantics (that
%         works for all values of the immediate operand) for those instructions
%     because that way is more intuitive and easy to maintain.  

%  \item \textbf{Un-Handled \emph{undef} flags} There are instructions which
%  conditionally sets some cpu flags to undefined state (henceforth referred as \emph{undef} state). For example, the shift left
%  instruction \instr{salq \%cl, \%rbx} sets flag \reg{of} to \emph{undef} state
%  if the count mask $>1$.  Also there are instructions like \instr{andnq \%rax,
%   \%rbx, \%rcx} which un-conditionally puts flags like \reg{pf} \& \reg{af} to
%   \emph{undef} state. \Strata does not attempt to learn the output relation
%   for undefined values that are placed in output registers, nor allow
%   instructions to read from an undefined location.
     
    % \item \textbf{Un-Handled \reg{af} flag} 

      
    % \item \textbf{Unreliable\revisit{??} Generalization to Immediate and Memory} How
    % faithful is the the hypothesis of ``generalization'' of register
    % instructions to memory or immediate variants?  
    
    % \item \textbf{Complex Formula} \Strata works in a stratified fashion in the
    % sense that simple instructions are learned first, followed by slightly more
    % complex instructions that reuse the entire formula of the simpler instructions learned previously.
    % This
    % implies that the formulas learned later  in the process might become very complex; even beyond the simplifications rules that \Strata apply later. For example, the \Strata formula for \instr{shrxl \%edx, \%ecx, \%ebx} contain around
    % $8971$\cmt{$2560$} terms (including the operator symbols) after simplification. 
    
    % Another interesting observation about \Strata learned artifacts is: For a non-floating point instruction \s{I}, \Strata might learn an instruction sequence containing floating point instruction(s). This might be possible because the instruction sequence happens to agree with \s{I} on a set of test inputs.  Now a symbolic execution on that instruction sequence might  yield a formula containing  those floating point operations expressed as \uif{}. Note that this will happen only when the simplification rules are not able to
    % eliminate the redundant floating point operation. For example, the \Strata
    % formula for \instr{vpbroadcastb \%xmm1 \%ymm2} contain  instances of uninterpreted function, \s{add\_double(0, A)} (representing addition of two double precision floating point values),
    %  whereas the instruction semantics itself has nothing to
    % do with floating point operations. Naive simplification rules  like $\s{add\_double(0, A)}$ $\equiv$ \s{A} cannot eliminate such redundant \uif{}   unless we apply
    % some additional heuristics to make sure that \s{A} is not $-0.0$.
    
    % In all cases, we aim to have simpler and more intuitive formula specifications.
    
%     \item \textbf{Different Artifacts for Execution \& Formal Reasoning: } As mentioned in section~\ref{sec:prelimstrata}, \Strata provides two different artifacts as the semantics of an instruction; First, an instruction sequence which  can be  executed  on real hardware and hence can be used as an executable artifact. 
%     Second artifacts are SMT formulas generated by 1) Symbolically executing the  instruction sequence , and 2) Later simplification using various simplification lemmas. The resulting SMT formulas provide an important artifact for formal reasoning about \ISA programs. Unfortunately, these formulas cannot be used for execution on actual machine  because of the presence of \uif. In that sense we have different artifacts used for different purposes and the question that alarms us is: Can we trust those formulas given the fact that it cannot be tested? More specifically, Can we trust the symbolic execution and the later simplification stages?       
% \end{enumerate} 

% \paragraph{Addressing the Challenges imposed by Strata Semantics}
%      Next we will elaborate how we mitigate all the challenges mentioned
%      above. 

%   \begin{enumerate}
    %   \item \textbf{Getting Concise Formula for Immediate Variants: }
    %   The goal is to get a single concise formula for those immediate instructions for which \Strata
    %   learned separate formulas for each constant value of the immediate operand (provided the
    %   immediate operand size  is  $8$-bits). We  obtain a candidate
    %   for such a concise formula by writing its semantics
    %   specification manually in \K and then use APIs provided by \K framework
    %     to get the required SMT formula. While doing so we consulted the Intel
    %   Manual~\cite{IntelManual} and tested the specification using the testing
    %   infrastructure mentioned in section~\ref{sec:Eval}.  
    %   We then compare each \Strata learned formula, for a constant say \s{c}, with the concise formula (that we just obtained) by replacing its symbolic inputs  with the constant value
    %   \s{c}. A successful equivalence check, for all values of \s{c}, suggest our concise formula
    %   to be of same correctness guarantee that \Strata has for any of the individual
    %   separate formulas. 
       
    %   Also, for some immediate instructions, \Strata learns a subset of $256$ separate formulas. 
    %   For each constant, say \s{z}, for which \Strata do not have a separate formula, we test our concise formula against the actual hardware after replacing its symbolic inputs  with the constant value \s{z}.
       

       
       
    %   \item \textbf{Modeling `af' Flag}
    %   \cmt{
    %   \begin{figure}[t]
    %       \centering
    %       \scalebox{0.7}{
               
    %           \incfig{figures/af_distribution.eps}
    %       }
    %       \caption{Instructions affecting \reg{af} flag.} The numbers represent count of Instruction variants. 
    %       \label{fig:AD}
    %   \end{figure}
       
    %   Figure \ref{fig:AD} represents the distribution of instructions
    % }There are total $689$\footnoteref{note1} instructions affecting the \reg{af} flag either in a defined or undefined  way.  Also the undefined-ness can occur
    %   conditionally or unconditionally. For all those 
    %   instructions, we ensure that the `af' is modeled faithful by consulting
    %   the Intel Manual. We tested the behavior of `af'-flag against actual hardware similar to the testing of \udef flags.

%   \item \textbf{Establishing Faithfulness of ``Generalization''} We validated
%   the the semantics of  memory or immediate instruction variants which are
%   obtained from the register variant using ``generalization'' by co-simulating
%   it  against a physical x86 machine.  While doing that,  we found
%   instructions like \instr{movsd (\%rax), \%xmm0} where the
%   generalization from the corresponding register variant, \instr{movsd \%xmm1, \%xmm0} is not faithful. Followings are their specification as per the Intel manuals.
   
%   {\small 
%       \begin{center}
%       \begin{tabular}[b]{l}
%           // Memory Variant: movsd  (\%rax) , \%xmm0 \\
%           {\tt S1.} XMM0[63:0] $\leftarrow$ MEM\_ADDR[63:0] \\
%           {\tt S2.} XMM0[127:64] $\leftarrow$ 0 \\ \\
           
%           // Register Variant: movsd \%xmm1  , \%xmm0 \\ 
%           {\tt S1.} XMM0[63:0] $\leftarrow$ XMM1[63:0] \\ 
%           {\tt S2.} XMM0[127:64] (Unmodified) \\
%       \end{tabular}
%     \end{center}
%   }   
%   Clearly, the memory variant cannot
%   be obtained using generalization of the  corresponding register instruction. Same is true with instruction \instr{movss (\%rax), \%xmm0}.   

   
   

    % \item \textbf{Ensuring Single Artifact for Both Execution and Formal Reasoning: }
    % The problem with \Strata is that artifact which can be used for formal reasoning is non-executable owing to the presence of \uif. 
    % We have implemented the semantics of most of the \uif{} using the Java bindings~\cite{MPFRJAVA} of GNU MPFR library~\cite{GNUMPFR}. 
    % Empowered by the fact that now we can use a single artifact for both execution and formal reasoning, we tested all \K rules which are either derived from  \Strata or manually written. To our surprise we found bugs in the simplification lemmas in project \Strata (details are given in section ~\ref{sec:Eval}).  

%   \end{enumerate}

% \subsection{Modeling Unsupported Instructions}\label{sec:US}
% For the remaining \currentManual{} which are not supported by \Strata, we manually defining their semantic directly in \K by consulting the Intel
% manual~\cite{IntelManual} and testing the model extensively.



\section{Validation of Semantics} \label{sec:Eval}

A formal semantics is of limited use if one cannot generate confidence in its correctness. In this section, we describe how we establish that confidence in our model. 


%\Comment{Sandeep: Add subsection here on the parser: it's important.}


\subsection{Co-Simulations against Hardware}
Empowered by the fact that we can directly execute the semantics using the \K framework,  we validated our model by  co-simulating  it against a real machine. During co-simulation, we execute a machine program on the processor as well as on our \K model and compare the execution results after every instruction. \SC{In this work, we co-simulated our model against two Intel implementations that were available to the authors at the time of writing: ``Intel Xeon CPU E3-1505M v6'' and ``Intel Xeon CPU E5-2640 v4''. We admit that testing the model against other hardwares (such as AMD) would contribute to more thorough validation of our model, having the potential of revealing flaws in those implementations and/or additional imperfections in the manual as well, which we leave as future work.}

We first describe our test-infrastructure and then talk about individual validation experiments and results.


\paragraph{Test Harness}

During co-simulations, we need to make sure that the program must be instrumented similarly both on our model and the real hardware. We use the GNU Debugger~\cite{GDB} to instrument programs on hardware. We developed instrumentation tools based on \K framework to gain similar capabilities for our model. Using these tools we can record the output state (including memory) after the execution of each instruction. 
To facilitate debugging, in the event when the output states do not match, we developed a tool which points to the first instant when the output states diverge and this saves debugging time. 

The co-simulation experiments are done in the following two phases: (1) Instruction level validation: testing individual instructions, and (2) Program level validation: testing combination of instructions  as a part of real-world programs. 

\paragraph{Instruction Level Validation}
The goal here is to execute individual instructions both on hardware and our model using test inputs and then compare the output states. 

\K already has matured library support for bit-vector, integer and floating point theories. \SC{We use bit-vectors to implement the values stored in registers or memory.} Depending upon an instruction mnemonic, these values can be interpreted as integers (signed/unsigned) or floating point values (with various precisions).  We augmented the library support in \K framework to interpret these bit-vectors accordingly. With that support, we can execute and hence test instructions implementing various floating point operations including conversions (to and from integer/floating point values) with selectable rounding modes (e.g. Nearest, +Inf, -Inf and Truncate).   

\subparagraph{Test Inputs} %6630 + 67 (ROL mcsema) + 200 (FP)
A test input is a CPU state which includes values for all registers, flags and memory. Our test input set contains more than $7,000$ inputs, obtained from the following sources: 
\begin{itemize}
    \item In section~\ref{sec:prelimstrata}, we mentioned that \Strata starts its algorithm with a set of test inputs which keeps on augmenting itself during the process of stratification. We used the final augmented test-suite of  $6630$ test inputs.
    
    \item While testing instructions implementing floating point operations, we found that many of the test inputs are representing a NaN or Infinity and it does not makes sense to test with many instances of these. We did our best effort by manually generating more than $100$ unique floating point values by  consulting the IEEE floating point arithmetic standard~\cite{FP}.
    \item We used the (${\sim}100$) test-inputs offered by Remill~\cite{Remill}.
    \item We manually implemented a regression test-suite worth of around $200$ test-inputs which we accumulated over the course of the project. 
\end{itemize}  
\SC{Note that, each instruction semantics consists of one or more semantic rules, where those rules cover different cases of the instruction behaviors (including the undefined ones). We ensure that our test inputs are sufficient enough to trigger all of the semantic rules, achieving the full ``semantic-rule'' coverage.}

\cmt{The extended test-inputs are tried with different combinations to test individual instructions and we ensured that all the instruction are tested with test inputs unique to the combination of its operands.}   

\subparagraph{Results}
For each immediate instruction with a constant operand of size $8$-bits, we tested all the $256$ variants of the instruction using the above set of test inputs. There are $62$ immediate instructions  with a constant operand width larger than $8$-bits.  Testing with all possible values of the constant (which could be $2^{32}$ for a $32$-bit constant) is impractical, so we limited the constant operand to the first $256$ values and other interesting values like  all ones, setting or resetting the bits at the byte/word/quad-word boundaries etc.  

Our current implementation of the fused-multiply-add operation\footnote{According to the standard IEEE-754-2008~\cite{FP} (Definition 2.1.28), the operation \s{fused-multiply-add(x, y, z)} computes $\s{x} \times \s{y} + \s{z}$ as if with unbounded range and precision, rounding only once to the destination format.} incorrectly rounds the operation twice (after multiplication and addition) as opposed to once. As a result, we encountered floating point precision issues while testing instructions implementing those operations (e.g. \instr{vfmadd132pd}).
\SC{This is a limitation of the underlying \K library and more details about this limitation can be found at Section~\ref{sec:limit}.}% of our floating point library support in \K. 
 
 
While performing the validation tests, we encountered various cases where the output state obtained by executing the semantics on our model does not agree with that of the hardware execution. 
The instruction semantics in our model is either based on the Strata project (for the part we borrowed) or on the Intel manual. A difference in the output state could mean a bug in Strata's instruction semantics or in our interpretation of the Intel manual or in the Intel manual itself. We found many bugs in our interpretation which we fixed, but in other cases, we found issues  in Intel manual and Strata project.

\subparagraph{Inconsistencies Found in the Intel Manual} Here are inconsistencies found during development and testing. \cmt{This is unsurprising given that fact that there are over $3,800$ pages of technical content and the vast majority of which is not machine-checked.} 
\begin{itemize}
    \item According to the manual, the semantics of \instr{vpsravd} \instr{\%xmm3, \%xmm2, \%xmm1} seems to depend on the lower $100$ bits of \reg{xmm3}, whereas the actual hardware execution suggest that it should depend on the lower $128$ bits. Similar inconsistencies are found in instructions with mnemonics \instr{vpsllvd}, \instr{vpsllvq}, \instr{vpsravd}.
    
    \item We found misleading typos related to instructions with opcodes \instr{vpsravw}, \instr{vpsravd}, \instr{vpsravq}, \instr{packsswb}.    
\end{itemize}
All these items were reported and acknowledged by Intel as issues in the manual~\cite{Suppl}
 
\subparagraph{Inconsistencies Found in \Strata's Simplification Rules}
While testing the instructions specifications borrowed from \Strata, we found inconsistent behaviors with the actual hardware. Moreover, the inconsistencies were discovered in the formulas of floating point instructions. This is not surprising because \Strata models the floating point instructions as \uif{} which cannot be executed or tested on hardware. Their semantics are executable in our definition though, and thus we were able to test them thoroughly. Note that Strata generates the formulas for these instructions by symbolically  executing  the corresponding learned  instruction sequences followed by a formula simplification pass. Therefore, errors in those formulas can be due to bugs either in the symbolic execution engine or in the simplification stage. Our testing shows that the second is true with the following evidence:
\cmt{  
\Strata generates , which we used as our baseline semantics, by 
All the instruction sequences are testing as part of the their algorithm, but the SMT formulas does not seem to be tested because of the   
 Further debugging unravels that the inconsistencies are due to following bugs in the simplification rules of \Strata, used to simplify the \Strata generated formulas.  
}
  \begin{itemize}
      \item The simplification rule \s{add\_double(A, 0) == A} does not hold for \s{A = $-0.0$}. Same for \s{add\_single}. These were reported~\cite{PC1}.
       
      \item The simplification rule \s{sub\_double(A, A) == 0} does not hold for \s{A = $NaN$}.  Same is true for \s{sub\_single}. We found this bug in the  branch of Stoke which is used in \Strata. But this has been already fixed in the latest \Stoke branch.
  \end{itemize}

\paragraph{Program Level Validation}
The goal here is to test the combination of instructions as part of real-world programs and we chose to use GCC C-torture tests~\cite{CTORTURE} for this purpose. Specifically, we used the tests inside the ``testsuite/gcc.c-torture/execute'' directory for GCC version 8.1.0.  
There are originally \TortureTotal{} tests, which we compiled using the GCC switches ``\s{-O0}  \s{-march=haswell} \s{-S} \s{-mlong-double-64} \s{-mno}-\s{80387}''. The last two switches avoid generating x87 instructions that are not in the scope of work. We had to exclude \TortureExclude{} programs containing system-level instruction \instr{prefetchnta}, which require modeling caches, which we currently do not support. Many test-cases involve C-library functions, like \s{malloc, fprintf}, most of whose semantics are modeled in \K. As our support of C-library functions is not exhaustive, we have to exclude $22$ programs containing un-supported functions like \s{vfprintf} and \s{vsprintf}, which we plan to support in future. This brings us to a grand total of \TortureInclude{} viable tests, which are all tested. Out of those, we found that there are \TortureUifsInstr{} cases where floating point instructions are used covering \TortureUifs{} unique floating point operations. Moreover, all the test-cases together cover about \TortureCoverage{} instruction variants, covering  $30\%$ of our supported instructions. As before, we executed each program on the processor as well as  on our model and compared the output state after every instruction, which matches in all the cases\footnote{Note that none of test-cases include floating point instructions implementing fused-multiply-addition, which we already acknowledged to have precision issues.}. 


\subsection{Comparing with \Stoke}
\SC{\Stoke~\cite{Stoke2013}}\footnote{\SC{Recall that \Stoke is a stochastic super-optimizer  leveraged by Strata for stochastic search.}} contains manually written semantics for \stokeIS{} x86-64 instruction variants, a large fraction ($81\%$) of which is also supported by \Strata. The remaining fraction is exclusive to Stoke. Comparing with \Stoke provides an additional crosscheck on our model.  Moreover, these manually written formulas are based on a similar model of the CPU state to ours, which makes it easier to compare them against ours by using an SMT solver. While doing so we found inconsistencies between the two formalisms in a total of 16 mnemonics (42 instruction variants), and after careful analysis, identified these as errors in the \Stoke specification of instruction semantics:
\begin{enumerate}
    \item For instructions like \instr{addsubpd \%xmm1, \%xmm2} , the order of addition and subtraction specified by \Stoke is opposite to the one specified in the Intel Manual. Same is true with the mnemonic \instr{addsubps}. (Found in $12$ instruction variants.)
    
    \item  The instruction \instr{pslld \%xmm1, \%xmm2} implements a logical left shift of packed data by a count specified in \reg{xmm1}. \Stoke's specification vectorized the operand \reg{xmm1} which is incorrect according to the manual. Similar issues were found in instructions implementing the logical right shift operations on packed data. (Found in  $18$ instructions.)
    
    \item Instructions \instr{cvtsi2sdl \%eax, \%xmm1} \& \instr{vcvtsi2sdl} \instr{\%eax, \%xmm0, \%xmm1} are respectively SSE- and AVX-versions of the instruction to  convert a double-word ($32$-bit) integer to a scalar single-precision floating-point value. According to the manual, in the AVX-version,  the  destination bits $127-64$ of the  register \reg{xmm1} are updated to the corresponding bits in the first source operand \reg{xmm0}. This  is in contrast to the SSE-version of the instruction where the destination bits $127-64$ should remain unmodified. \Stoke specifies the semantics of the AVX-version similar to the SSE-version, which is incorrect.  (Found in $4$ instruction variants.)
    
    \item Some instructions, like \instr{imulb \%al}, which drive flag registers to an undefined state are not modeled correctly in \Stoke.    (found in $8$ instruction variants)  
\end{enumerate}
All these errors were reported through a pull request~\cite{Suppl}.

\cmt{ 

The goal here is to test the combination of instructions as part of real-world programs and we chose to use GCC C-torture tests~\cite{CTORTURE} for this purpose. Specifically, we used the tests inside the ``testsuite/gcc.c-torture/execute'' directory for GCC version 8.1.0.  
There are originally \TortureTotal{} tests, which we compiled using  the GCC switches ``\s{-O0 -march=haswell -S -mlong-double-64 -mno-80387}''. The last two switches are used to avoid generating X87 instructions for  $21$ cases. We had to exclude \TortureExclude{} programs, because those use at least one of the following: (1) GCC specific extensions or builtins like \s{\_\_builtin\_add\_overflow}; (2) C library functions, like \s{malloc, printf}, which we do not support; (3) system-level instructions like \instr{prefetchnta}; or (4) assembler directives like \s{.comm, .string and .bss}, which we currently do not support. This brings us to a grand total of \TortureInclude{} viable tests. Out of those, we found that there are \TortureUifsInstr{} cases where floating point instructions are used covering \TortureUifs{} unique floating point operations. Moreover, all the test-cases together cover about \TortureCoverage{} instruction variants, covering  $30\%$ of our supported instructions. As before, we executed each program on the processor as well as  on our model and compared the output state after every instruction, which matches in all the cases\footnote{Note that none of test-cases include floating point instructions implementing fused-multiply-addition, which we already acknowledged to have precision issues.}.


\revisit{?? }
In the process we
found that the semantics of most of those flags are already available in
\Stoke. We needed to model the semantics of flag registers for
additional $40$ instructions involving {\tt shifts,
    rotates}~\cite{BugStoke986}.
\revisit{undef modelling}This help
finding bugs in the \Stoke implementation of $8$
instructions~\cite{BugStoke986}.
\revisit{af flag model}
In the process we implement the semantics of that
flag for $40$ instructions~\cite{BugStoke986} (these are not supported
in \Stoke). The `af'-flag semantics for the rest of the instruction
are obtained from the \Stoke project.
}

\section{Applications} \label{sec:Appl}

In this section, we illustrate a few applications of our formal semantics, in addition to the reference model mentioned in the previous section.
Our goal here is to demonstrate that our semantics can be used for formal reasoning of \ISA programs for a wide variety of purposes.
For this reason, the applications are illustrative only and do not represent a comprehensive evaluation. Moreover, for each application, we simply used the tools which \K framework provides out of the box and do not work towards improving their runtime, which is not very impressive at time of writing this paper.
However, we believe that each application, with our preliminary evaluation, can be leveraged into a \emph{practical} standalone tool, with its own user interface, case studies and even reference manual, but this is not our goal here.
In fact, thanks to the language-parametric nature of \K, none of these reasoning approaches can be regarded as novel {\it per se}, because they are already used in the context of other languages defined in \K and their implementation is language-semantics agnostic.
%
We begin with a discussion of a use case for hardware verification.

% All these applications are realized based on the  generic tools offered by the employed \K semantic framework.

% \Qd{Is the placement of application Ok..or we need to place the sym ex example before verification and say something about progressing from simpler to more complex capability}
\cmt{
These application tools are
automatically derived from the semantics and changes made to the
semantics immediately affect the tools. This is as opposed to writing analysis tools 
independently of semantics, which could be error prone and/or difficult to maintain. 
}

%----------------------------------------------------------------------------------------------------------
\subsection{Validating Processor Hardware and Documentation}
\label{sec:Appl:Processor}
%----------------------------------------------------------------------------------------------------------

Verification is considered one of the most (if not the most) important 
challenges in modern processor design, for several reasons: 
(i) the enormous state spaces of modern systems; 
(ii) the lack of formal specifications in the state-of-practice, 
(iii) generating high quality test inputs for simulation, 
(iv) quantifying/analyzing the extent of coverage of simulation, and 
(v) generating a complete set of properties for checking. 
For post-silicon validation, an additional challenge is the difficulty 
of debugging and diagnosing observed erroneous behaviors. 
For all these reasons, verification is estimated to use 70\% of the 
resources and time, while design takes only 30\%~\cite{Foster:DAC2015}.

A fully executable formal ISA-level specification such as the one developed here can improve the state of practice in verification in two significant ways. 

First, it can provide a reliable specification of functional behavior of 
hardware with respect to observable states. This increases 
confidence in the input tests, for both directed and random test generation. 
High confidence tests can reduce time and increase focus during debugging, 
triage and diagnosis efforts. This is especially valuable in post-silicon 
validation, where observability within the chip is very limited, and functional 
validation is a key goal. This method can also help post-silicon failure 
diagnosis by identifying buggy input/output pairs and pinpointing specific 
erroneous output state bits.

Second, since our method can symbolically execute instructions, it can be used 
to generate input tests that have high coverage. While such analyses have been 
done at the detailed RTL level \cite{liu2014, liu2011, mishra18}, we are 
unaware of similar tools at the \ISA ISA level\footnote{It is worth mentioning the work by Martignoni et al. ~\cite{Martignoni:ASPLOS2012} about test-case generation for $32$-bit x86 ISA.}. The most significant advantage of 
such symbolic execution is the ability to detect corner case or hard to detect 
bugs~\cite{liu2014,Fonseca:Eurosys2017}.  (This is analogous to finding security vulnerabilities 
due to corner-case software bugs, illustrated in 
Section~\ref{sec:Appl:Security}, but applied to the hardware implementation instead of software.)  We expect that ISA level symbolic analysis will 
uncover such subtle and complex bugs due to the higher level of abstraction 
and better design perspective than the RTL analyses.

A closely related challenge is checking the accuracy of ISA specifications, including reference manuals.
%
By using such manual specifications to construct a formal specification, we may uncover errors in the manual specifications.
%
This is explicitly demonstrated by the two bugs we discovered in the Intel \ISA manual, while performing the instruction-level validation tests described in Section~\ref{sec:Eval}.
%
These bugs were discovered as a result of running test cases using both the formal semantics generated by reading the manuals and the hardware, and finding a mismatch, then checking the manual specification  carefully to determine whether the bug lies in the manuals or in the hardware.
%
However, given an existing semantics, a far more valuable strategy would be to \emph{automatically generate human-readable documentation from the formal specification}.
%
A basic version of this strategy is likely quite feasible today, and much more sophisticated versions that synthesize illustrative examples and even explanatory text automatically could be possible soon, given recent advances in concolic test generation, program synthesis, and natural language processing.

\cmt{In recognition of the many benefits of an executable formal specification, Intel has an internal project to develop such a specification of the \ISA ISA semantics.}
%
%We are in communication with a team in Intel interested in such a specification, and plan to contribute in future~\cite{Intel:PC2}.

%----------------------------------------------------------------------------------------------------------
\subsection{Program Verification}
\label{sec:Appl:Verification}
%----------------------------------------------------------------------------------------------------------

\begin{figure}
% \centering
% \begin{tabular}{l@{\hskip 0.25in}|@{}l}
\begin{lstlisting}[
language=C++, 
basicstyle=\scriptsize,
keywordstyle=\color{blue},
%numberstyle=\tiny\color{codegray},
%numbers=left,
%xleftmargin=\parindent
]
  int s = 0; int n = N;
  while (n > 0) { s = s + n; n = n - 1; }
  return s;
\end{lstlisting}
\vspace{-5pt}
\begin{center}
{\small (a) C source code}
\end{center}
% & 
\begin{lstlisting}[
language=Bash,
keywordstyle=\color{blue},
morekeywords={movl,cmpl,jle,addl,decl,jmp,ret},
commentstyle=\color{gray},
basicstyle=\scriptsize,
%numberstyle=\tiny\color{codegray},
breaklines=true,
breakatwhitespace=false,
%numbers=left,
%xleftmargin=\parindent,
stepnumber=1
]
  movl %edi, -20 (%rbp)   # %edi holds N
  movl $0, -4 (%rbp)      # s = 0
  movl -20(%rbp), %eax
  movl %eax, -8(%rbp)     # n = N
  L3: # loop header  
    cmpl $0, -8(%rbp)     # check n <= 0
    jle L2  # if n <= 0, then jump to end, else continue
    movl -8(%rbp), %eax   # n > 0 at this point
    addl %eax, -4(%rbp)   # s = s + n
    decl -8(%rbp)         # n = n - 1
    jmp L3                # jump back to loop header
  L2:
    movl -4(%rbp), %eax   # n <= 0 at this point
    ret
\end{lstlisting}
\vspace{-5pt}
\begin{center}
{\small (b) \ISA assembly code}
\end{center}
% \multicolumn{1}{c|}{(a)} & \multicolumn{1}{c}{(b)} \\ 
% \multicolumn{1}{c|}{\footnotesize C version} & \multicolumn{1}{c}{\footnotesize \ISA assembly version} \\ \hline
% \end{tabular}
\vspace{-5pt}
\caption{\s{sum-to-n} program}
%\vspace{-20pt}
\label{fig:sum2n}
\end{figure}



The \K framework provides a language-parametric, reachability logic theorem prover~\cite{Stefanescu:2016,Rosu:2012}.  We instantiated it with our semantics to generate a correct-by-construction deductive verifier for \ISA programs.
Here, the functional correctness properties are specified as reachability specifications, essentially a pair of pre- and post-conditions for each function.
The derived \ISA verifier uses a sound and relatively complete proof system to prove the given specifications w.r.t.~the \ISA semantics.
Like in other deductive verifiers, repetitive constructs such as loops and recursive functions need to be annotated with invariants.
The verifier is automatic: it requires only the program, its specification, and the invariants.

% K offers support for program verification based on rule-based
% semantics, at no additional cost (with no need to define another
% semantics)~\cite{Rosu:2012}. Program properties are specified as reachability
% rules. K uses a sound and relatively complete proof system for
% deriving such rules from the operational semantics rules, which
% amounts to:
% \begin{enumerate}
%     \item Performing symbolic execution of code without repetitive behavior
%     using the semantics rules; and
%     \item Reasoning about repetitive constructs (loops, recursion). 
% \end{enumerate}

% Like in Hoare logic, all the repetitive constructs need to be annotated
% with specifications. The verification is automatic: the user only provides
% the specifications. The specifications are given as reachability
% rules between symbolic configurations with constraints 

To demonstrate that our semantics can be used to verify \ISA programs, we use the \ISA verifier to prove the functional correctness of the \s{sum-to-n} program as shown in Figure~\ref{fig:sum2n}.
It takes $N$ as input and returns the sum from 1 to $N$.
% The assembly code of the program, \fig{fig:sum2n}(b), is annotated with comments to show its resemblance with the C code, \fig{fig:sum2n}(a).
The functional correctness can be essentially described as: $\reg{rax} = \sum_1^N n = N (N + 1) / {2}$.
%
We present the actual specification that is fed to the \ISA verifier.
The specification has two parts: the top-level specification and the loop invariant.

% To test the viability of using the generic reachability verification
% infrastructure with the \ISA semantics, we verified the functional correctness of a  simple  ``sum from $1$ to $n$'' program as shown in Figure~\ref{fig:sum2n}.  We use this program as a demonstration of the capability of our model to reason high level properties about programs. The assembly language version  of the program, \fig{fig:sum2n}(b), is annotated with comments to show its resemblance with the C-version on the left, \fig{fig:sum2n}(a).

% The high level property that we intend to proof is that the at program point 26 the value of \reg{rax} is given by: 
% $$\reg{rax} = \sum_{i = 1}^N i = \frac{N * (N + 1)}{2}$$




% Next,  we present the \K specification rules uses to proof the the property. It consists of  two reachability rules: (1) Main rule stating the functional correctness of the program as a whole; and (2) The  loop invariant rule stating the functional correctness of the loop.

% Note that the program behaves incorrectly/unexpectedly if arithmetic overflow occurs during its execution. One challenge in verifying this program is to identify the conditions under which overflow does not occur. 

\begin{figure}[ht]
\begin{lstlisting}[style=KRULE]
<regstate>... 
    "RDI" (*$\mapsto$*) 64'N
    "RBP" (*$\mapsto$*) 64'56
    "RIP" (*$\mapsto$*) (64'0 => 64'-1) 
    "RAX" (*$\mapsto$*) (64'_ => 64'(N * (N + 1)) / 2)
...</regstate>
<memstate>...
  // -8(%rbp): n
  48 (*$\mapsto$*) (byte(0,_) => byte(0, 32'0))
  49 (*$\mapsto$*) (byte(0,_) => byte(1, 32'0))
  50 (*$\mapsto$*) (byte(0,_) => byte(2, 32'0))
  51 (*$\mapsto$*) (byte(0,_) => byte(3, 32'0))
  // -4(%rbp): s
  52 (*$\mapsto$*) (byte(0,_) => byte(0, (32'(N * (N + 1)) / 2)))
  53 (*$\mapsto$*) (byte(0,_) => byte(1, (32'(N * (N + 1)) / 2)))
  54 (*$\mapsto$*) (byte(0,_) => byte(2, (32'(N * (N + 1)) / 2)))
  55 (*$\mapsto$*) (byte(0,_) => byte(3, (32'(N * (N + 1)) / 2)))
...</memstate>
    requires N >= 0 and N < 2^31
         and (N * (N + 1)) / 2 < 2^31
\end{lstlisting}
\begin{center}
\vspace{-5pt}
{\small (a) Top-level specification
%\Comment{s is a signed int, so result should be less than $(2^{31} -1)$?  Or make s unsigned in C source.}
}
\vspace{-5pt}
\end{center}
% \caption{Specification of \s{sum-to-n} program}
% \label{fig:main-rule}
% \end{figure}
% \begin{figure}[t]
\begin{lstlisting}[style=KRULE]
<regstate>... 
    "RIP" (*$\mapsto$*) (L3 => L2)
...</regstate>
<memstate>...
  // -8(%rbp): n
  48 (*$\mapsto$*) (byte(0, A) => byte(0, 32'0))
  49 (*$\mapsto$*) (byte(1, A) => byte(1, 32'0))
  50 (*$\mapsto$*) (byte(2, A) => byte(2, 32'0))
  51 (*$\mapsto$*) (byte(3, A) => byte(3, 32'0))
  // -4(%rbp): s
  52 (*$\mapsto$*) (byte(0, B) => byte(0, 32'(B + A * (A + 1) / 2)))
  53 (*$\mapsto$*) (byte(1, B) => byte(1, 32'(B + A * (A + 1) / 2)))
  54 (*$\mapsto$*) (byte(2, B) => byte(2, 32'(B + A * (A + 1) / 2)))
  55 (*$\mapsto$*) (byte(3, B) => byte(3, 32'(B + A * (A + 1) / 2)))
...</memstate>
    requires A >= 0 and A < 2^31
         and B >= 0 and B < 2^31
         and B + ((A * (A + 1)) / 2) >= 0  
         and B + ((A * (A + 1)) / 2) < 2^31
\end{lstlisting}
\begin{center}
\vspace{-5pt}
{\small (b) Loop invariant}
\vspace{-5pt}
\end{center}
% \caption{Loop Invariant Rule. X and Y denote concrete code memory addresses corresponding to instructions at loop header (at line 10) and following the loop (at line 25) resp.}
% \label{fig:loop-invariant}
\caption{Specification of \s{sum-to-n} program}
\label{fig:sum-to-n-spec}
\end{figure}




% \paragraph{Top-Level Specification}

Figure~\ref{fig:sum-to-n-spec}(a) shows the functional correctness specification of the \s{sum-to-n} program.
The \s{regstate} cell specifies the relevant registers used in the program, omitting the irrelevant ones denoted by ``...''.
Specifically, it specifies that \reg{rdi} holds the value $N$ without being updated during the program execution, and \reg{rax} will have the expected return value.
The \s{memstate} cell specifies the relevant part of the memory omitting others (denoted by ``...'').
It specifies the stack memory addresses \s{-8(\%rbp)} and \s{-4(\%rbp)} corresponding to \m{n} and \m{s}, respectively.
The \m{requires} clause specifies the condition of $N$ that prevents the arithmetic overflow.
% The program counter \reg{rip} starts from $0$ (the code memory address) and ends at the return address (which we specify using $-1$).
% The first part of the rule is largely static, which means that the contents of configuration cells are not going to change during execution and is given by the \K specifications as follows:
% which essentially represents the initial values of \reg{rdi} and \reg{rbp} and the fact that these values will remain the same throughout the program execution.
%
% The second part consist of the main claim states the following
% \begin{itemize}
%     \item The progarm counter \reg{rip} starts from $0$ (the code memory address corresponding to the instruction at line 2) and ends at the return address( which we specify using $-1$).
%     \item The stack memory address \s{-4(\%rbp)}, which intends to store the value of sum, starts with ``don't care'' state  and ends with the correct sum. 
%     \item The stack memory address \s{-8(\%rbp)}, which intends to store the value of $n$ for comparison purpose, starts with ``don't care'' state  and ends with $0$. 
%     \item The value of $N$ is such that overflow will not occur during execution.     
% \end{itemize}
% and is specified using the following \K rules:
% \paragraph{Loop Invariant}
Figure~\ref{fig:sum-to-n-spec}(b) shows the loop invariant specification.
It specifies the behavior of an arbitrary loop iteration.
That is, assuming the values of \m{n} and \m{s} be $A$ and $B$, resp., in the beginning of an arbitrary loop iteration, it specifies their final values in the end of the entire loop execution, which are 0 and $B + A(A+1)/2$, respectively.
Note that when $A = N$ and $B = 0$, i.e., the first loop iteration, the loop invariant captures the entire loop behavior.

% The static part of this rule is same as the main claim.  
% The non-static part specifies the behavior of the program any time it reaches the loop head (line 8, Figure~\ref{fig:sum2n}). The behavior can be  specified informally as:
% \begin{itemize}
%     \item The program counter starts and ends at code memory addresses corresponding to instruction at line 10 and  25 respectively.
%     \item The stack memory address \s{-4(\%rbp)}, which intends to store the value of sum, starts with a partial sum \s{B}  and ends with the correct sum.   
%     \item The stack memory address \s{-8(\%rbp)}, which intends to store the value of $n$ for comparison purpose, starts with non zero positive value \s{A}  and ends with $0$.  
%     \item \s{A} and \s{B} are sufficiently low that overflow will not occur during execution.
% \end{itemize}
% and  in terms of \K specifications as:




\SC{The \K verifier takes a minute}\footnote{%
\SC{We note that the reported verification time would not be a major concern for scalability because of the modularity of deductive verification.}
} to verify the \m{sum-to-n} assembly code satisfies the functional correctness specification.

% The overall verification took approx. 60 secs. We believe that our preliminary evaluation delivers a realistic potential of using the semantics for \ISA program verification. \revisit{about the library/lemmas to assst reasoning about bitvectors}



%\subsection{Finding Security Vulnerabilities}
\subsection{Symbolic Execution}
\label{sec:Appl:Security}

\K automatically derives a correct-by-construction symbolic execution engine from the given semantics.
Being instantiated with our semantics, the engine can be used to symbolically execute and explore all possible paths in the given \ISA program.
In this section, we demonstrate how this capability can be used to find a  security vulnerability.
% In practice, this capability would be used more efficiently than exhaustive path exploration by using Dynamic Symbolic Execution~\cite{godefroid_automated_2008}.

\begin{figure}
% \centering    
% \begin{tabular}{l@{\hskip 0.25in}|@{}l}
\begin{lstlisting}[
language=C++,
basicstyle=\scriptsize,
keywordstyle=\color{blue},
breaklines=true,
breakatwhitespace=false,
morekeywords={uintptr_t,uint64_t},
commentstyle=\color{gray},
numbers=left,
xleftmargin=\parindent
]
uintptr_t safe_addptr(int *of, uint64_t a, uint64_t b) {
  uintptr_t r = a + b;
  if (r < a) { // "error state" 
    *of = 1; return r;
  } else {     // "safe state" 
    return r;
  }
}
\end{lstlisting}
\vspace{-10pt}
\begin{center}
{\small (a) C source code}
\end{center}
\begin{lstlisting}[
language=Bash,
keywordstyle=\color{blue},
morekeywords={movl,cmpl, jle, addl, decl, jmp, ret, sbbl, jnc},
commentstyle=\color{gray},
basicstyle=\scriptsize,
%numberstyle=\tiny\color{codegray},
breaklines=true,
breakatwhitespace=false,
numbers=left,
xleftmargin=\parindent,
stepnumber=1,
escapeinside={(*}{*)}
]
  # Address %ebp - 8  contains 64-bit value 'a'  
  # Adress  %ebp - 16 contains 64-bit value 'b'
  # Let a[32:0]: lower 32 bits of 'a' 
  #     b[32:0]: lower 32 bits of 'b'
  
  movl -8(%ebp), %edx   # a[32:0] moved to %edx
  movl -16(%ebp), %eax  # b[32:0] moved to %eax
  addl %edx, %eax       # r = a[32:0] + b[32:0]
  movl $0, %edx         # check if (32'0 (*$\circ$*) r) < a
  cmpl -8(%ebp), %eax   #  (*$\circ$*): denotes concatenation
  movl %edx, %eax
  sbbl -4(%ebp), %eax
  jnc L2                # true branch: "error state"
  ...                   # set *of to 1 and %eax to r
  jmp L3
L2:                     # else branch: "safe state"
  ...                   # set %eax to r
L3:
  ret
\end{lstlisting}
\vspace{-15pt}
\begin{center}
{\small (b) \ISA assembly code in a 32-bit target}
\end{center}
% \multicolumn{1}{c|}{(a)} & \multicolumn{1}{c}{(b)} \\ 
% \multicolumn{1}{c|}{\footnotesize C version} & \multicolumn{1}{c}{\footnotesize Assembly version} \\ \hline
% \end{tabular}
%\vspace*{-10pt}
\caption{A security vulnerability in the HiStar kernel}
\label{fig:histar}
%\vspace*{-15pt}
\end{figure}



Consider the code snippet of the HiStar~\cite{HiStar:2006} kernel, as shown in Figure~\ref{fig:histar}(a), in which the KLEE~\cite{Cadar:2008} team found a security vulnerability.
The \s{safe\_addptr} function is supposed to compute the sum of two arguments \m{a} and \m{b}, setting the flag argument \m{of} when the arithmetic overflow occurs during the addition.
That is, one of the functional correctness properties is that ``$\m{*of} = 1$ if $\m{a} + \m{b} > \m{r}$'', where $+$ is the mathematical addition (with no overflow).
The functional correctness, however, is not satisfied when the source code is compiled to a 32-bit target,
% as shown in Figure~\ref{fig:histar}(b),
since the size of \m{r} becomes 32-bit (\m{uintptr\_t}) while the sizes of \m{a} and \m{b} are still 64-bit (\m{uint64\_t}).\footnote{The function call \s{safe\_addptr(*of, address, size)} is used to validate that an user is allowed to access the memory range specified by the arguments \s{address} and \s{size}.
The access is denied if an overflow occurs.
A bug in the overflow detection might be exploited by an attacker to gain an access to a memory region beyond the control of the running process.}
A suggested fix~\cite{Cadar:2008} is to change the conditional expression from \m{r < a} to \m{r < a || r < b}.

Using the symbolic execution engine derived from our semantics, we could find (in approx. $85$ secs) that, in the assembly code as shown in Figure~\ref{fig:histar}(b), there exists a path that reaches \m{L2} (i.e., the else branch) even if the addition overflow occurs.
The (simplified) path condition provided by the symbolic execution engine is:
\[
\m{a} + \m{b} \ge 2^{32}
~~\wedge~~
(\m{a} + \m{b} \mod 2^{32}) \ge \m{a}
\]
where $0 \le \m{a} < 2^{64}$ and $0 \le \m{b} < 2^{64}$.
We asked Z3 to solve the above path condition and it returned a solution (i.e., a concrete input to trigger the security vulnerability): \m{a} = \m{0x00000000ffffffff} and  \m{b} = \m{0xffffffff00000000}.



% TODO: we can add this if we have enough space.
% \begin{figure}[t]
% \begin{lstlisting}[style=KRULE,numbers=left, numberstyle=\tiny\color{codegray}]
% <regstate>...
%     "RIP" |-> (mi(64, 0) => L2)
%     "RBP" |-> (mi(64, 56))
% ...</regstate>
% <memstate>... 
%   // -8(%rbp): a
%   40 |-> (byte ( 0 , mi(64, B)))
%   41 |-> (byte ( 1 , mi(64, B)))
%   42 |-> (byte ( 2 , mi(64, B)))
%   43 |-> (byte ( 3 , mi(64, B)))
%   44 |-> (byte ( 4 , mi(64, B)))
%   45 |-> (byte ( 5 , mi(64, B)))
%   46 |-> (byte ( 6 , mi(64, B)))
%   47 |-> (byte ( 7 , mi(64, B)))
%   // -16(%rbp): b
%   48 |-> (byte ( 0 , mi(64, A))) 
%   49 |-> (byte ( 1 , mi(64, A))) 
%   50 |-> (byte ( 2 , mi(64, A))) 
%   51 |-> (byte ( 3 , mi(64, A))) 
%   52 |-> (byte ( 4 , mi(64, A))) 
%   53 |-> (byte ( 5 , mi(64, A))) 
%   54 |-> (byte ( 6 , mi(64, A))) 
%   55 |-> (byte ( 7 , mi(64, A))) 
% ...</memstate>
%     requires  A >= 0 and A < 2 ^Int 64 
%         and   B >= 0 and B < 2 ^Int 64 
%     // check if the oerflow condition is true.
%     ensures   A + B >= 2 ^ 32
% endmodule
% \end{lstlisting}
% \caption{Specification used to do symbolic execution on code snippet at Figure 7.}
% \label{lst:CSHISTAR}
% \end{figure}




% symbolically executed the assembly code
% in Figure~\ref{fig:histar}(b) to explore all possible paths and find 



% Symbolic execution is straightforward to achieve using \K framework because exactly the  same semantics rules applied for both concrete  and symbolic execution. While doing symbolic execution, the applied semantics rules constraint the symbolic terms which are then solved using \Z~\cite{DeMoura:2008} (which is incorporated in K). 

% \cmt{ 
% written o work with concrete terms
% using a rewriting-based semantics:
% whether a term is concrete or abstract makes no difference to the
% theory. Rules designed to work with concrete terms needs no change in order to work with symbolic terms. The symbolic terms being mathematical variables could be constrained.
% }


% while $\m{a} + \m{b} \ge 2^{32}$ where $+$ is the mathematical addition.


% While the functional correctness is satisfied when the source code is compiled to a 64-bit target, it is not for a 32-bit target.


% The security bug is found and reported by the developer's of KLEE~\cite{Cadar:2008} while doing symbolic execution on $32$-bit of the HiStar~\cite{HiStar:2006} kernel.



% Figure~\ref{fig:histar}(a) shows the C source code of the problematic function \m{safe\_addptr}.

% Figure~\ref{fig:histar}(a) presents the code for the function containing the bug. The function \s{safe\_addptr}
% \footnote{\s{safe\_addptr(*of, address, size)} is used to validate if user is allowed to  access a memory range as specified by the arguments `address' and `size'. In case of overflow, access is denied. A bug in the overflow detection check might be exploited by an attacker to gain access to memory regions beyond the control of running process.}
%     \cmt{ 
%      function is used for validating the user memory address by taking as argument an address and a size in order to compute if the user is allowed to access the memory in that range. The routine uses the overflow to prevent access when a computation could overflow. Failure to do so might allows a malicious process to gain access to memory regions outside its control.} is supposed to check if the addition overflows. If true, then the control reaches ``error state'' and return after setting \s{*of} to true. Otherwise, control reaches ``safe state'' and return.
%      The function arguments being $64$-bit long, the test condition used is not sufficient (it should be (r < a) || (r < b)) and can fail to indicate overflow for large values of b. 

% Figure~\ref{fig:histar}(b) shows a $32$-bit compiled version of the assembly program   corresponding to the C-code on the left. The assembly code is simplified for brevity reasons. Our approach to unravel the bug is to try driving the program to a state such that 1. The program counter is at ``safe state'', and 2. The addition result  overflows. If we are successful, then we found the bug. Listing~\ref{lst:CSHISTAR} is the specification for the same; Line 5  specifies  the first requirement and line 33 ensures the second.  



% A symbolic execution on the code snippet at \fig{fig:histar} drives us to the required state demanded by the specification above. Upon solving the accumulated constrains (at program point corresponding to ``safe state'') using \Z, we can get a concrete input, $\mathtt{a = 0x00000000ffffffff}$ and  $\mathtt{b = 0xffffffff00000000}$, which can trigger the bug.



\subsection{Translation Validation of Optimizations}


\begin{figure}[t]
\begin{lstlisting}[language=C++,basicstyle=\scriptsize,keywordstyle=\color{blue}]
  int popcnt(uint64_t x) {
    int res = 0;
    for (; x > 0; x >>= 1) { res += x & 0x1ull; }
    return res;
  }
\end{lstlisting}
\caption{\s{popcnt} program}
\label{fig:popcnt}
\end{figure}


\K also provides a program equivalence checker that can be used for the translation validation of compiler optimizations.
We derived an \ISA program equivalence checker from our semantics and used it to validate different optimizations.
Figure~\ref{fig:popcnt} shows a program that we considered, \s{popcnt}, which counts the number of set bits in the given input.

We compiled the program with different optimizations: the GCC compiler optimizations (-O0, -O1, -O2, and -O3), and the Stoke super-optimization.
On top of the baseline (-O0), the -O1 optimization produces a code obtained by performing the \s{mem2reg} optimization, the -O2 optimization produces one by factoring out the common statement over different branches,\footnote{Specifically, by performing the common subexpression elimination, followed by certain statement reordering optimization, followed by the strength reduction.} and the -O3 optimization produces the same code with -O2. The Stoke super-optimization translates the assembly code into a single instruction: \instr{popcnt \%rdi, \%rax}, where \reg{rdi} and \reg{rax} correspond to the input and the return values, respectively.

We validated these optimizations by checking the equivalence between the optimized programs.
The equivalence checker symbolically executes each program and compares their return values (i.e., the symbolic expression of the \reg{rax} register value) using Z3.
It is able to prove successfully that all optimization variants are equivalent, i.e., to check the correctness of all these optimizations on \s{popcnt}.

Note that the symbolic execution of the \s{popcnt} program, which took approx. ~$20$ mins, does not require an additional annotation about the loop because the number of loop iterations is bound to a constant (i.e., the bit-size of the input, 64).
In general, however, the equivalence checker may require us to provide an additional annotation about loops, which can be automatically generated by augmenting the underlying compiler.




% \subsection{Program Equivalence Checker across Optimization}

% The aplication we showcase here is to proof equivalence of \ISA programs across optimizations. For presentation, we chose a program (\fig{fig:countone}(a)) which counts the number of set bits in a $64$-bit integer.
% \begin{figure}[]
\centering
\begin{tabular}{l|l}
\begin{lstlisting}[language=C++, basicstyle=\scriptsize,keywordstyle=\color{blue}]
int popcnt(uint64_t x) {
  int res = 0;
  for ( ; x > 0; x >>= 1 ) {
    res += x & 0x1ull;
  }
  return res;
}
\end{lstlisting}

& 
{\begin{lstlisting}[style=KRULEWOBORDER]
// lshrMInt(BitVector, BitVector) 
//    (*$\rightarrow$*) BitVector
//  A function to evaluate 
//  bit-vector logical right shift.
// W'V 
//  A bit-vector of width W 
//  and unsigned value V.    
rule lshrMInt(
  lshrMInt(W'V, W'V1), W'V2) 
    (*$\Rightarrow$*) lshrMInt(W'V, W'V1 + W'V2)
  requires 
    V1 (*$\leq$*) W and V2 (*$\leq$*) W
    
rule lshrMInt(W'V1, W'V2) 
    (*$\Rightarrow$*) W'0
  requires V2 (*$=$*) W
\end{lstlisting}} \\ 
\multicolumn{1}{c|}{(a)} & \multicolumn{1}{c}{(b)} \\ 
\multicolumn{1}{c|}{\footnotesize C Program} & \multicolumn{1}{c}{\footnotesize  Lemmas for loop reasoning} \\ \hline
\end{tabular}
\caption{Example showing Program Equivalence Checking Capability.}
\label{fig:countone}
\end{figure}


% We compiled the program using different GCC optimizations \s{-O$0$/-O$1$/-0$3$} (The output of \s{-0$2$} and \s{-O$3$} are exactly  same) and symbolically execute the resulting \ISA programs using symbolic value of \s{x}. Note that the reasoning about the loop termination, which took $24$ secs, is fully automatic and is facilitated by \K's matured bit-vector theory support. Figure~\ref{fig:countone}(b) shows the relevant lemmas used for loop reasoning. \footnote{As evident from the lemmas that the loop will be iterated $64$ times before termination and hence the reason of longer time.  Runtime can be improved by adding loop invariants.} 

% Using \Z, we pairwise matched the SMT formula corresponding to \reg{rax} (used as program return value) which is accumulated at the end of the execution. As expected, all the formulas are equivalent. Also, we employed \Stoke super-optimizer to further optimize the program generated by GCC (\s{-O3}), which produces the optimal rewrite with \instr{popcnt \%rdi, \%rax}. We used symbolic execution as above to  prove its equivalence to the other variants.        



\subsection{Limitations}\label{sec:limit}
Our limitations mostly include missing features of the \ISA and execution environment.
%
\begin{itemize}
 \item {\em Floating Point Operations\/}: Our testing shows that we have FP precision issues with instructions implementing the fused-multiply-add operation. \SC{This is because the current \K's floating point library~\cite{MPFRJAVA} implementation lacks support of the FMA capabilities of GNU MPFR library~\cite{GNUMPFR}, which we plan to include in future.}
 \item {\em Exceptions\/}: We do not support exceptions, including the FP exceptions. Moreover, we do not distinguish between quiet and signaling NaN, i.e. all NaNs are quiet in our model. \SC{When the exception condition is encountered, execution proceeds after setting the exception flag.}
 \item {\em Concurrency\/}: Like the closest previous work~\cite{Goel:FMCAD14,Heule2016a}, we do not model concurrent semantics or the relaxed memory model. Other previous work has defined a formal, executable semantics for the x86 memory model \cite{Sarkar:POPL09,Owens:x86-TSO}, and we leave it future work to merge that with our semantics. The design of our model  is parameterized by the memory model and hence amenable to plug-in another one.
 \item {\em Instruction Decoding\/}: The instruction decoding semantics is \emph{not} modeled in the current work. However, we want to note that we have formalized the Intel XED disassembler algorithm, and we are working on formalizing the instruction decoding.
\end{itemize}

\cmt{ 
Missing functionality in the \K tools limits our ability to run some programs:
%
\begin{itemize}
 \item {\em C-standard library support\/}: \K does not yet support the C-library, which is necessary for widely used operations like \s{malloc, printf}. We are actively working with the \K team on modeling those library functions.
 %
 \item {\em Assembler directives\/}: Our parser does not yet support certain assembler directive like \s{.comm, .string and .bss}; we plan to support these in future.
\end{itemize}
}

\section{Related Work}\label{sec:RW}

% \Qt{Suggestion: I do not think we should include coverage numbers of projects that do not support direct semantics. And here are the reasons:
%     1. Projects like angr or mcsema do have support  of x87 or mmx which we do not have. That way if we include 
%     x87/mmx in the total instructions count we cannot say that we are 100\%.
%     2. Now if we exclude x87/mmx instructions and give the percentage (and that way we can say we are 100\%), but  that may confuse people. 
% 
%     We can include Strata and Goel's work in the table (or write in text ) and *can* include the percentages because neither of them support x87/mmx. For the others we can argue that they are not direct. Also we may 
%     add that bap, remill and radare2 are not complete. I can later will in what exact instruction they are missing  
%  }
% 
% \Qd{\revisit{I think we should ommit exact percentage numbers all together in the table. We should replace that column with a simple yes/no checkmark whether the semantics is complete. Also remove all together projects that do not give semantics directly to x86 but to some IL.}}

%There have been many projects that host a formal semantics of \ISA either as
%their main contribution or as part of their infrastructure.
%Table~\ref{table:RW} summarizes such previous work and compares it to our formal semantics.\footnote{Here we focus on comparing with other \emph{direct} semantics, since a complete \emph{direct} semantics is our goal and required for our purpose. We will discuss other \emph{indirect} semantics later in this section.}
%We do the comparison in three directions that reflect the
%primary contributions of our work: the completeness of the definition in terms
%of supported user-level instructions, the faithfulness of the definition in
%terms of whether it is executable and hence can be evaluated with real code
%execution, and the generality of the definition in terms of its applicability to
%formal reasoning analyses. Next, we discuss in more detail each of the
%related works.

There have been many projects that host a formal semantics of \ISA either as
their main contribution or as part of their infrastructure.
This section summarizes such previous work and compares it to our formal semantics based on three directions that reflect the primary contributions of our work: the completeness of the definition in terms of supported user-level instructions, the faithfulness of the definition in
terms of whether it is executable and hence can be evaluated with real code
execution, and the generality of the definition in terms of its applicability to
formal reasoning analyses. 

\cmt{
\begin{table}
\scalebox{0.7}{
%\setlength{\arrayrulewidth}{.15em} 
\begin{tabular}{l||ccc}
\hline
\\ [-10pt]
\multicolumn{1}{l||}{\begin{tabular}[]{@{}c@{}}Project \\ Name\end{tabular}} & 
\multicolumn{1}{c}{\begin{tabular}[c]{@{}c@{}}Complete Support of \\ \ISA User-Level \\ Instructions (in scope) \end{tabular}} &
\multicolumn{1}{c}{\begin{tabular}[c]{@{}c@{}}Executable \\ Semantics\end{tabular}} &
\multicolumn{1}{c}{\begin{tabular}[c]{@{}c@{}}Support for \\ Full-Fledged \\ Formal Reasoning\end{tabular}} \\
\hline 
\\ [-10pt]
Strata~\cite{Heule2016a}          & \xmark & \rating{50} & \xmark      \\
Goel et al.~\cite{Goel:FMCAD14}  & \xmark & \cmark      & \cmark      \\
CompCert~\cite{Leroy:2009}        & \xmark & \cmark      & \cmark      \\
%Remill~\cite{Remill}              & \xmark & \cmark      & \xmark      \\
TSL~\cite{TSL:TOPLAS13}           & \xmark & \cmark      & \rating{50} \\
Sail ~\cite{sail-framework}        & \xmark & \cmark      & \cmark \\
Roessle \etal~\cite{Roessle:CPP19} & \xmark & \rating{50} & \cmark \\
\hline
\textbf{Our Semantics}            & \cmark & \cmark      & \cmark      \\
\hline
\end{tabular}}
\begin{center} 
{\small
    \cmark : Yes
    \quad
    \xmark : No % due to incorrect semantics
    \quad
    \rating{50} : Partially True
    \hfill
    %\quad
    %\emph{NJ}: Node.js 0.10.29
}
\end{center}
\caption{Projects hosting formal semantics of the \ISA ISA.}
\label{table:RW}
\end{table}
}

\Strata~\cite{Heule2016a} uses program synthesis to generate the instruction
semantics, as SMT bit-vector formulas, by learning their input/output behavior
through execution on an actual processor. \Strata's semantics definition for
\ISA is incomplete as it does not include semantics of about 40\% of the
user-level instructions in scope that are not straightforward to automatically learn by the presented algorithm for reasons mainly due to limitations of underlying synthesis engine. Even then,
the result that the formal semantics of 60\% of the target \ISA ISA can be learned automatically
is impressive, and we leverage this result in our work as described in
section~\ref{sec:harvestsema}.
The specifications are executable only for non floating-point (FP) instructions.
The FP operations are represented in the SMT formulas of the definition as
uninterpreted functions. Finally the specifications are given as SMT formulas
but have not been demonstrated to be usable in a formal analysis setting out-of-the-box.

\SC{A contemporary work by Roessle \etal~\cite{Roessle:CPP19} presents a method to extract the big step semantics of a sequential binary program using the small step instruction semantics extracted mostly from Strata\footnote{There are some minor omissions on immediate instructions with $8$-bit operands for which Strata learns $256$ brute force formulas.} plus some manually drafted support for branching and stack operating instructions. Like Strata, their specification is executable only for the non-floating point instructions. However, the work does not aim for completeness of semantics, which is one of our primary goals.} 

Goel \etal~\cite{Goel:FMCAD14} use the ACL2 theorem prover~\cite{ACL2:Kaufmann2000} to model the \ISA ISA and has support for
\goelPerc{} of all user level instructions~\cite{GoelList}. They have specifications of a selection
of user-level instructions, some system-level instructions, paging, and
segmentation. This list is far from a complete semantic definition of \ISA,
however it is still the state-of-the-art in terms of formal analysis applied
directly to \ISA code. It is also an executable definition as demonstrated in
its use for simulations. In our work we do not leverage this definition, since
we found a more complete definition such as \Strata to be a better starting
point towards completeness.

The CompCert verified compiler~\cite{Leroy:2009} includes semantics
definitions for all intermediate and target languages used within the compiler,
including a definition for x86 assembly. The definition is specified in Coq~\cite{Coq}, and has been used in a formal
setting for proving the correctness of the CompCert's compilation step to assembly,
as well as outside CompCert, e.g., in proofs relating to the certified concurrent
OS kernel CeriKOS~\cite{Gu:2016}. However, this definition focuses on the
32-bit x86 instruction set, which is a subset of the \ISA instruction 
set.
Moreover, it is part of the trust base for CompCert and it is not clear
whether or how it has been tested against an actual processor, whereas
\Strata and ours have been extensively tested.

TSL~\cite{TSL:TOPLAS13} is a system that can auto-generate tools for
various machine code analyses given a semantics definition of the machine
language written in the TSL specification. Such a semantics
definition for the integer instructions (i.e., no floating-point instructions) of the $32$-bit x86 instruction set is given
as part of the project. It is used for the generation of
various tools, including a machine code synthesizer~\cite{Srinivasan2015}.
This definition, to the best of our knowledge, have not been used
for formal specification proofs, i.e., to prove whether a given x86
program meets its specification.

Our semantics, like all of the aforementioned works, uses a sequential consistency memory model, omitting weaker memory models.
Existing efforts to specify weaker memory models for \ISA such as Owens et al.~\cite{Owens:x86-TSO} and Sarkar et al.~\cite{Sarkar:POPL09}, however, suffer from their limited support for instruction semantics (i.e., they consider only a small subset of 32-bit x86 instruction set).
We believe that integrating these two complementary efforts is a promising direction toward rigorously reasoning about real-world programs running on modern multiprocessors (e.g., using the Sail framework as we will describe below).

\SC{%
Sail is another language semantics framework, tailored for describing an instruction-set architecture semantics.  Sail has been used to specify the semantics of ARMv8-A, RISC-V, and CHERI-MIPS~\cite{sail-popl2019}, as well as the semantics of a small subset of x86~\cite{sail-x86}.  Sail is similar to the K framework we employed, but K is for more general-purpose than Sail.  Also, the Sail x86 semantics is very limited than ours.  It describes the semantics of a fragment of 32-bit user-mode x86 instructions, while ours covers also the 64-bit counterpart as well as the floating-point instructions.
%
Sail, however, allows us to integrate a semantic definition with their relaxed memory models~\cite{rmem, Pulte:2017} for concurrency semantics.  We believe that (automatically) translating our semantics into Sail\footnote{Indeed, the Sail ARMv8-A semantics is automatically generated from the ARM-internal specification of ARMv8-A~\cite{Reid2017} written in the ARM's architecture specification language, ASL~\cite{asl}, by using the ASL-to-Sail translator~\cite{sail-popl2019}.} is a promising direction to obtain concurrency semantics and thus enable concurrency reasoning for x86 programs, which we leave as future work.
}

\SC{On the other hand, the key differentiator of our effort compared to the above mentioned existing work is that it combines (A) completeness of supported  user-level instructions, (B) faithfulness, and (C) applicability to formal reasoning analyses, at the same time. In Section~\ref{sec:lesson-learned}, we will share the key takeaways from our work, which paid us off towards achieving the unique combination.}

There are various binary analysis projects that target \ISA binaries
and lift them to a higher-level representation more suitable for the
specific analysis. These include Angr~\cite{Angr} using the VEX IR of Valgrind~\cite{Valgrind:ENTCS03}, the QEMU~\cite{QEMU:USENIX05} emulator
using the TCG IR, the software fault isolation tool RockSalt~\cite{Roclsalt:PLDI12} using its own RTL DSL, the disassembler and binary analyzer Radare2~\cite{Radare2} using the ESIL IR~\cite{ESIL}, the binary analysis
tool BAP~\cite{BAP:CAV11} using the BIL IR, and  the static binary translator Remill~\cite{McSema:Recon14} using LLVM IR~\cite{LLVM:CGO04}.
We refer these semantics as \emph{indirect} as they give the semantics of the \ISA binary via the translation to their IR, as opposed to the \emph{direct} semantics such as ours.
% because in all these cases the lifted IR, that is being analyzed, is rigorously defined as opposed to providing \emph{direct} semantics to the binary.
Note that the direct semantics has its own value.
For example, without the direct semantics of \ISA, we cannot even formulate the correctness of a translator from \ISA to the IR.
Analogously, many programming languages (C, C++, Java, etc.) have been given direct semantics, instead of indirect semantics by translation to other languages, for formal reasoning at the desired language granularity. 

% Even though tools like BAP and Angr can do some formal reasoning owing to their capability of symbolically executing the IR semantics, but they are not designed with the goal of full-fledged formal reasoning.




\cmt{ 
Regarding the comparison with previous work, we focused on comparing with other direct semantics, since a complete *direct* semantics is our goal and required for our purpose.

In all these
cases the IR that is being analyzed is rigorously defined, but we refrain from
considering these as formal specifications of \ISA because the
actually specified language has abstracted away various features of \ISA.
For example, both VEX IR and RockSalt use different simplified register
semantics: VEX IR omits many implicit bit truncations
and/or extensions that are part of many \ISA instruction semantics 
(i.e., these have to be emulated separately by the program), while
RockSalt's DSL uses an infinite register file instead of the finite
\ISA register file.
}

Hasabnis \etal~\cite{Hasabnis:ASPLOS16, Hasabnis:FSE16} also present an indirect semantics of \ISA, but in contract to other indirect semantics, they use machine learning~\cite{Hasabnis:ASPLOS16} and symbolic 
execution~\cite{Hasabnis:FSE16} to automatically learn the translation of \ISA instructions to their IR, by extracting knowledge from the hard-coded  translation logic of compilers such as GCC.
However, as they admitted~\cite{Hasabnis:FSE16}, their semantics omits some important details of \ISA semantics (e.g., the effect of various instructions on CPU flags), and thus is not sufficient to serve as a solid foundation for rigorous formal analyses of \ISA binary.

  
\cmt{ 
Remill~\cite{McSema:Recon14} is a static binary translator from \ISA to LLVM
IR~\cite{LLVM:CGO04}. The translator contains specifications for \ISA instructions
in the form of equivalent C\cmt{LLVM IR} programs to assist the translation. This
specification is neither complete nor formal and cannot be easily used
in formal analysis.
}


% implement
% lifting of \ISA to an architecture-independent intermediate language (IL).
% In contrast with the other works above,

% They learn these mappings by extracting
% knowledge from the hard-coded translation logic found in compilers such as GCC.
% The extracted mappings cover more than 99\% of \ISA instructions. However,
% \cmt{similar to the rest of binary analysis works,}
% the resulting IL cannot stand as a formal \ISA definition because it
% abstracts away important \ISA semantic details, e.g.,
% the effect of the various instructions on CPU flags.




\section{Lessons Learned}
\label{sec:lesson-learned}
Here we present the lessons we learned during our semantics development,
     identifying important aspects to be considered, and clarifying a good
     practice for developing a large ISA semantics.  We also elaborate the
     novel aspects of our semantics development approach that allow us to
     obtain the complete and faithful semantics with an affordable amount of
     effort.

\paragraph{Regarding automatic semantics synthesis}

Section~\ref{sec:Approach:Overview} reports the empirical negative result that a fully automatic synthesis of the whole \ISA semantics is difficult to achieve.
Specifically, in a vast instruction set like \ISA \cmt{computer (CISC) architecture such as x86}, it is common that many instructions can be grouped together where the instructions of each group are similar to each other except for a few differences.
%One approach to automatic semantics synthesize is to employ stratification approach~\cite{Heule2016a} to learn the semantics of instruction variants in a group using the representatives of the group.
An automatic synthesis technique, using the stratification approach~\cite{Heule2016a}, would effectively synthesize such instruction variants' semantics, provided that the semantics of its representative instructions are given in advance.
\footnote{For certain complex instructions, the size of their group is very small (i.e., they are quite different to each other), and thus the automatic synthesis would not yield a sufficient gain over the effort of specifying the semantics of their representatives, but we found that the number of such isolated instructions of x86-64 is small.}
The problem is, however, that it is non-trivial to properly partition all the instructions into such groups, providing the representative instruction semantics for each group, \emph{without} a priori knowledge about the semantics of all instructions.
%The stratification approach~\cite{Heule2016a} had been proposed to solve this dilemma, but it turned out to be not sufficient, leaving a substantial part of semantics unspecified.
The vanilla stratification approach~\cite{Heule2016a} turned out to be not sufficient to solve this dilemma, leaving a substantial part of semantics unspecified.
Thus, we decided to manually provide the information about partition and representatives, for which we had to consult the manual to obtain knowledge about the remaining part of semantics.
Once we obtained the required knowledge, however, we realized that it would be more straightforward to directly turn the knowledge into the semantics than going through the synthesis process, and thus we ended up manually specifying the remaining part of semantics in this work.
%
This (negative) experience led us to think about a new research problem: how to obtain the prior knowledge about the semantics that is enough to partition instructions and identify their representatives, \emph{without} manually examining the entire manual.
This  knowledge does not need to be precise, and we believe that such rough information can be automatically extracted from the manual using text processing.  We plan to explore this idea while defining the semantics of unspecified and/or new instructions.
%We suggested an early idea for this problem, which we plan to develop and evaluate further as future work (Section~\ref{sec:conc}).

%As mentioned in Section~\ref{sec:Approach:Overview}, we found that our ideas of extending \Strata do not scale because we need information about the instructions, whose semantics we want to learn, in order to constraint the search space. This lesson suggested a new idea: the information needed to reduce the search space can be automatically extracted from the manual. 
%This  information does not need to be precise, and we believe that such rough information can be automatically extracted from the manual using text processing.  We plan to explore this idea while defining the semantics of unspecified and/or new instructions.

%As an alternative, we suggest a hybrid synthesis approach that we believe is practically promising, in particular for complex instruction set computer (CISC) architectures such as x86.
%
%While it is labor-intensive and error-prone to manually specify all the instruction variants, the automatic synthesis technique is shown to be effective to synthesize such instruction variants' semantics, provided that the semantics of its representative instruction is given in advance.
%
%Thus, in a hybrid approach, the semantics of representative instructions are manually written, and their variants are automatically synthesized. We believe that this hybrid approach is in a sweet spot, where the machine's search power and the human reasoning are effectively combined.
%We plan to apply and evaluate this idea when we specify the semantics of new instruction that will be introduced in the next version of x86-64 ISA.

Most of the previous efforts in formalizing \ISA semantics can be categorized based on whether the underlying approach is fully manual ~\cite{Goel:FMCAD14, TSL:TOPLAS13, Leroy:2009, sail-x86} or fully automatic~\cite{Heule2016a, Roessle:CPP19, Hasabnis:ASPLOS16, Hasabnis:FSE16}. We note that none of these approaches, when used in isolation, sufficiently scale to a complete and faithful semantics, as much as ours that combines these complementary approaches and benefits from each other.

Another important step of the semantics synthesis is post-processing. The generated semantics is often verbose and not necessarily human-readable. The post-processing step is desired to simplify the generated semantics to be succinct, which helps to increase the human-readability as well as to improve the efficiency when being employed in other applications (e.g., the size of SMT formula encoding can be reduced, which can reduce the burden of SMT solvers). For our semantics development, we have written dozens of simplification rules that are fed to the \K framework to simplify the synthesized semantics further (Section~\ref{sec:Approach}). 

\paragraph{Modeling and executing implementation-dependent behaviors}

Like many other programming languages, the \ISA ISA standard admits implementation-dependent behaviors, that is, each processor implementation can freely choose specific behaviors (Section~\ref{sec:challenges-in-formalizing-x86}).
There are two natural, faithful ways of specifying the implementation-dependent behaviors.
One is to parameterize the semantics over the implementation dependent behaviors, and later instantiate it with a profile that describes specific behaviors taken by the processor of interest.
Another is to introduce non-determinism in the semantics.
We took the second approach in this work, since the implementation-dependent behaviors of x86 are quite limited and mostly localized (e.g., as shown in Figure~\ref{fig:andn-semantics}).

However, the non-determinism makes it non-trivial to execute (and validate) the semantics.
For example, in the hardware co-simulation, the output of hardware execution may vary depending on the underlying processors, while the non-deterministic semantics execution will randomly choose certain behavior, which may be different from the specific behavior implemented in a processor.
To mitigate this issue, for the hardware co-simulation, we symbolically executed the semantics where the non-deterministic behaviors are represented symbolically, so that the symbolic output captures all possible behaviors, and then we checked if the hardware output is matched by (an instance of) the symbolic output.

We note that faithfully modeling the implementation-dependent behaviors is necessary for the correctness of the semantics.
As mentioned in Section~\ref{subsec:compare-stoke}, however, Stoke~\cite{Stoke2013} does not faithfully model such behaviors, causing certain errors in their semantics that we revealed~\cite{BugStoke986}.
On the other hand, most of other existing \emph{direct} \ISA semantics employ an approach similar to the ones described above, faithfully modeling the implementation-dependent behaviors.
For example, Goel \etal~\cite{Goel:ProCoS17} models such behaviors using a constraint function which is guaranteed to be unique and non-deterministic, while they employ the aforementioned profiling approach for concrete execution.
TSL~\cite{TSL:TOPLAS13} makes both approaches available, from which their users can choose.

\paragraph{Employing multiple semantic engineering frameworks}

We found that employing multiple semantic frameworks is helpful. Specifically, we employed the two semantic frameworks, \K and Stoke, where we enjoyed all of their (executive) benefits that make it easier for us to write and validate the semantics, and utilize the semantics in various applications. For example, we wrote the semantics of certain complicated instructions (e.g., \instr{pcmpestri}, \instr{pcmpestrm}, and \instr{pclmulqdq}) in \K, as \K provides an easy way to specify behaviors with multiple cases, while Stoke would have required us to write a big nested if-then-else expression which is not amenable. For another example, we utilized Stoke to validate most of our instruction semantics as Stoke provides a infrastructure\footnote{Indeed, we contributed to their infrastructure as well~\cite{completing-stock,improving-stoke}.} for the hardware co-simulation, whereas we employed \K to validate the semantics of floating-point instructions as Stoke does not support executing floating-point operations while \K does.

In order to use the two frameworks interchangeably, we had to develop a translator that translates the semantics in one framework to another. To convince the correctness of the translation, we employed translation validation, where we verified equivalence between the original and the translated semantics for each instruction using the Z3 SMT solver. 

Employing multiple frameworks, with the validated translation, allowed us to utilize their exclusive features to efficiently specify (e.g., by using the well-designed specification language of one framework) and validate (e.g., by using the well-developed testing infrastructure of another framework) the semantics, which highly expedited our semantics development process, and thus significantly contributed to the completeness of our semantics.
%, not to mention the benefit of using exclusive formal analysis tools.}
%Whereas, other approaches (~\cite{Goel:FMCAD14,TSL:TOPLAS13,Leroy:2009,sail-x86,Heule2016a}) did not explore this mutual benefit to its full potential.
Moreover, employing multiple frameworks in general allows us to immediately benefit from all of their formal analysis tools, increasing the applicability of the semantics in various formal reasoning tasks.
Existing semantics development efforts (e.g., \cite{Goel:FMCAD14,Heule2016a}), however, %(~\cite{Goel:FMCAD14,TSL:TOPLAS13,Leroy:2009,sail-x86,Heule2016a})
employ a single framework without utilizing the potential of other frameworks, which otherwise might have improved completeness and/or faithfulness of their semantics with the same amount of effort.

%\cmt{
%We note that employing \Strata enables us to explore the near-completeness of its formalism, whereas \K framework provides many out-of-the-box formal analysis tools. The fact that we can use both the frameworks  interchangeably paid us off towards completeness and generality (in terms of using the semantics for formal analyses) of the formalism. On the other hand, other approaches (~\cite{Goel:FMCAD14,TSL:TOPLAS13,Leroy:2009,sail-x86})  did not explore this mutual benefit to its full potential\footnote{That, for some of these projects,  could be attributed to the fact getting a complete formal semantics is not the primary goal.}.

%\section{Future Work}
\label{sec:Future}

As mentioned in Section~\ref{sec:Approach:Overview}, we found our ideas of extending \Strata do not scale because we need information about the instruction, whose semantics we want to learn, in order to constraint the search space. This lesson intrigued us to a new idea, where the information needed to reduce the search space can be automatically extracted from the manual. This
 information does not need to be precise, and we believe that such a
rough information can be automatically extracted from the manual using text processing.  We plan to explore this idea while defining the semantics of
 unspecified and/or new instructions.

\section{Conclusion and Future Work}\label{sec:conc}

We have presented the most complete formal semantics about \ISA user-level instructions
to date, which has been thoroughly tested and demonstrated with possible applications in different formal analyses such as symbolic execution, deductive
verification, and translation validation.
Moreover, the \K framework enables us to represent a semantics as SMT theories,
which other projects can
leverage for their own purposes.

As mentioned in Section~\ref{sec:Approach:Overview}, we found that our ideas of extending \Strata do not scale because we need information about the instructions, whose semantics we want to learn, in order to constraint the search space. This lesson intrigued us to a new idea: the information needed to reduce the search space can be automatically extracted from the manual. This  information does not need to be precise, and we believe that such rough information can be automatically extracted from the manual using text processing.  We plan to explore this idea while defining the semantics of unspecified and/or new instructions.

%\SC{We plan to test our model against various x86-64 implementations (like AMD), which could uncover flaws in those implementations and/or additional imperfections in the manual.}

%\SC{Also, we aim to use our semantics for translation validation of the entire LLVM compiler back-end for X86-64, i.e., hundreds of thousands of lines of C++ code. Scalability is achieved by the modularity of the translation validation technique, where the verification of (small) sub-components are verified individually (hence can be massively parallelized) and their results are combined to obtain the final claim about the entire system.}

 

%\K  has success stories about defining the formal semantics of production languages like C~\cite{KC} and LLVM~\cite{KLLVM}. We can use that towards the benefit of binary lifters like ~\cite{McSema:Recon14, FCD} which lifts the binary code to LLVM IR for binary analysis and/or optimization. Having the semantics of both the source (\ISA) and the target language(LLVM) help in verifying the translation using symbolic reasoning and hence enhance trust in  the translation.

%\textit{Acknowledgements.}
%We warmly thank the \K team
%   for their technical support throughout the project.  Also we thank the Strata
%   developers for promptly confirming
%   our reported bugs and for answering all our questions in great detail.
   

%\input{acknowledgement.tex}


%% Acknowledgments
\begin{acks}
We warmly thank the \K team
   for their technical support throughout the project.  Also we thank the Strata
   developers for promptly confirming
   our reported bugs and for answering all our questions in great detail.
\end{acks}

%
% The next two lines define the bibliography style to be used, and the bibliography file.
\bibliographystyle{ACM-Reference-Format}
\bibliography{bibs/references.bib,bibs/modeling-X86-semantics.bib,bibs/bugs.bib,bibs/k.bib}


%% Appendix
%\appendix
%\section{Appendix}
%
%Text of appendix \ldots

\end{document}
