\section{Limitations}\label{sec:limit}
Our limitations mostly include missing features of the \ISA and execution environment.

\vspace{2pt}
\noindent\textbf{\emph{Floating Point Operations}}: Our testing shows that we have FP precision issues with instructions implementing the fused-multiply-add operation. \SC{This is because the current \K's floating point library~\cite{MPFRJAVA} implementation lacks support of the FMA capabilities of GNU MPFR library~\cite{GNUMPFR}, which we plan to include in future.}
%%

\vspace{2pt}
\noindent\textbf{\emph{Exceptions:}} We do not support exceptions, including the FP exceptions. Moreover, we do not distinguish between quiet and signaling NaN, i.e. all NaNs are quiet in our model. \SC{When the exception condition is encountered, execution proceeds after setting the exception flag.}
%%

\vspace{2pt}
\noindent\textbf{\emph{Concurrency:}} 
Like the closest previous work~\cite{Goel:FMCAD14,Heule2016a}, we do not model concurrent semantics or the relaxed memory model as defined by other previous work \cite{Sarkar:POPL09,Owens:x86-TSO}. Our design, being parameterized on memory model, is amenable to accommodate others', which we plan to achieve in future.
%Like the closest previous work~\cite{Goel:FMCAD14,Heule2016a}, we do not model concurrent semantics or the relaxed memory model. Other previous work has defined a formal, executable semantics for the x86 memory model \cite{Sarkar:POPL09,Owens:x86-TSO}, and we leave it future work to merge that with our semantics. The design of our model  is parameterized by the memory model and hence amenable to plug-in another one.
%%

\vspace{2pt}
\noindent\textbf{\emph{Instruction Decoding:}} The instruction decoding semantics is \emph{not} modeled in the current work. However, we want to note that we have formalized the Intel XED disassembler algorithm, and we are working on formalizing the instruction decoding.
%
%\begin{itemize}
% \item {\em Floating Point Operations\/}: Our testing shows that we have FP precision issues with instructions implementing the fused-multiply-add operation. \SC{This is because the current \K's floating point library~\cite{MPFRJAVA} implementation lacks support of the FMA capabilities of GNU MPFR library~\cite{GNUMPFR}, which we plan to include in future.}
% \item {\em Exceptions\/}: We do not support exceptions, including the FP exceptions. Moreover, we do not distinguish between quiet and signaling NaN, i.e. all NaNs are quiet in our model. \SC{When the exception condition is encountered, execution proceeds after setting the exception flag.}
% \item {\em Concurrency\/}: Like the closest previous work~\cite{Goel:FMCAD14,Heule2016a}, we do not model concurrent semantics or the relaxed memory model. Other previous work has defined a formal, executable semantics for the x86 memory model \cite{Sarkar:POPL09,Owens:x86-TSO}, and we leave it future work to merge that with our semantics. The design of our model  is parameterized by the memory model and hence amenable to plug-in another one.
% \item {\em Instruction Decoding\/}: The instruction decoding semantics is \emph{not} modeled in the current work. However, we want to note that we have formalized the Intel XED disassembler algorithm, and we are working on formalizing the instruction decoding.
%\end{itemize}
%
%Missing functionality in the \K tools limits our ability to run some programs:
%%
%\begin{itemize}
% \item {\em C-standard library support\/}: \K does not yet support the C-library, which is necessary for widely used operations like \s{malloc, printf}. We are actively working with the \K team on modeling those library functions.
% %
% \item {\em Assembler directives\/}: Our parser does not yet support certain assembler directive like \s{.comm, .string and .bss}; we plan to support these in future.
%\end{itemize}
