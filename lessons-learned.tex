\section{Lessons Learned}
\label{sec:lesson-learned}

{\color{blue}

Here we present lessons we learned during our semantics development, identifying important aspects to be considered, and clarifying a good practice for developing a large ISA semantics.

\paragraph{Regarding automatic semantics synthesis}

In Section~\ref{sec:Approach:Overview}, we reported the empirical negative result that a fully automatic synthesis of the whole x86 semantics is not practically feasible.
Specifically, in a complex instruction set computer (CISC) architecture such as x86, many instructions can be grouped together where the instructions of each group are similar to each other except for a few differences.
An automatic synthesis technique would effectively synthesize such instruction variants' semantics, provided that the semantics of its representative instruction is given in advance.\footnote{%
For certain complex instructions, the size of their group is very small (i.e., they are quite different to each other), and thus the automatic synthesis would not yield a sufficient gain over the effort of specifying the semantics of their representatives, but we found that the number of such isolated instructions of x86-64 is small.}
The problem is, however, that it is non-trivial to properly partition all the instructions into such groups, providing the representative instruction semantics for each group, \emph{without} a priori knowledge about the semantics of all instructions.
The stratification approach~\cite{Heule2016a} tried to solve this dilemma, but it was not sufficient, leaving a large part of semantics unspecified.
Thus, we decided to manually provide the information about partition and representatives, for which we had to consult the manual to obtain knowledge about the remaining part of semantics.
Once we obtained the required knowledge, however, we realized that it is more straightforward to directly turn the knowledge into the semantics than going through the synthesis process, and thus we ended up manually specifying the remaining part of semantics in this work.


%As an alternative, we suggest a hybrid synthesis approach that we believe is practically promising, in particular for complex instruction set computer (CISC) architectures such as x86.
%
%While it is labor-intensive and error-prone to manually specify all the instruction variants, the automatic synthesis technique is shown to be effective to synthesize such instruction variants' semantics, provided that the semantics of its representative instruction is given in advance.
%
%Thus, in a hybrid approach, the semantics of representative instructions are manually written, and their variants are automatically synthesized. We believe that this hybrid approach is in a sweet spot, where the machine's search power and the human reasoning are effectively combined.
%We plan to apply and evaluate this idea when we specify the semantics of new instruction that will be introduced in the next version of x86-64 ISA.


Another important step of the semantics synthesis is post-processing. The generated semantics is often verbose and not necessarily human-readable. The post-processing step is desired to simplify the generated semantics to be succinct, which helps to increase the human-readability as well as to improve the efficiency when being employed in other applications (e.g., the size of SMT formula encoding can be reduced, which can reduce the burden of SMT solvers). For our semantics development, we have written dozens of simplification rules that are fed to the \K framework to simplify the synthesized semantics further (Section~\ref{sec:Approach}).

\paragraph{Modeling and executing implementation-dependent behaviors}

Like many other programming languages, the x86-64 ISA standard admits implementation-dependent behaviors, that is, each processor implementation can freely choose specific behaviors (Section~\ref{sec:challenges-in-formalizing-x86}).
There are two natural, faithful ways of specifying the implementation-dependent behaviors.
One is to parameterize the semantics over the implementation dependent behaviors, and later instantiate it with a profile that describes specific behaviors taken by the processor of interest.
Another is to introduce non-determinism in the semantics.
We took the second approach in this work, since the implementation-dependent behaviors of x86 are quite limited and mostly localized (e.g., as shown in Figure~\ref{fig:andn-semantics}).

However, the non-determinism makes it non-trivial to execute (and validate) the semantics.
For example, in the hardware co-simulation, the output of hardware execution may vary depending on the underlying processors, while the non-deterministic semantics execution will randomly choose certain behavior, which may be different from the specific behavior implemented in a processor.
To mitigate this issue, for the hardware co-simulation, we symbolically executed the semantics where the non-deterministic behaviors are represented symbolically, so that the symbolic output captures all possible behaviors, and then we checked if the hardware output is matched by (i.e., an instance of) the symbolic output.

\paragraph{Employing multiple semantic engineering frameworks}

We found that employing multiple semantic frameworks is helpful. Specifically, we employed the two semantic frameworks, \K and Stoke, where we enjoyed all of their (executive) benefits that make it easier for us to write and validate the semantics, and utilize the semantics in various applications. For example, we wrote the semantics of certain complicated instructions (e.g., \instr{pcmpestri}, \instr{pcmpestrm}, and \instr{pclmulqdq}) in \K, as \K provides an easy way to specify behaviors with multiple cases, while Stoke would have required us to write a big nested if-then-else expression which is not amenable. For another example, we utilized Stoke to validate most of our instruction semantics as Stoke provides a infrastructure\footnote{Indeed, we contributed to their infrastructure as well~\cite{improving-stoke}.} for the hardware co-simulation, whereas we employed \K to validate the semantics of floating-point instructions as Stoke does not support executing floating-point operations while \K does.

In order to use the two frameworks interchangeably, we had to develop a translator that translates the semantics in one framework to another. To convince the correctness of the translation, we employed translation validation, where we verified equivalence between the original and the translated semantics for each instruction using the Z3 SMT solver.

}